\documentclass[
  fontsize=11pt,
  paper=landscape,
  twocolumn=true,
  pagesize=pdftex,
  headings=small,
  DIV=15,
  ]{scrartcl}
\usepackage{amssymb,amsmath,enumerate}
\usepackage[T1]{fontenc}
\usepackage[osf]{mathpazo}
\usepackage{inconsolata}

\usepackage{url}

\usepackage[unicode=true]{hyperref}
\hypersetup{breaklinks=true, pdfborder={0 0 0}}

\usepackage{listings}
\lstset{
  basicstyle=\small\ttfamily,
  breaklines=true
}

\setcounter{secnumdepth}{0}
\setlength{\columnsep}{2em}
\setkomafont{disposition}{\normalfont\bfseries}
\title{In the Beginning was the Command Line}
\author{Neal Stephenson}

\begin{document}
\maketitle
About twenty years ago Jobs and Wozniak, the founders of Apple, came up
with the very strange idea of selling information processing machines
for use in the home. The business took off, and its founders made a lot
of money and received the credit they deserved for being daring
visionaries. But around the same time, Bill Gates and Paul Allen came up
with an idea even stranger and more fantastical: selling computer
operating systems. This was much weirder than the idea of Jobs and
Wozniak. A computer at least had some sort of physical reality to it. It
came in a box, you could open it up and plug it in and watch lights
blink. An operating system had no tangible incarnation at all. It
arrived on a disk, of course, but the disk was, in effect, nothing more
than the box that the OS came in. The product itself was a very long
string of ones and zeroes that, when properly installed and coddled,
gave you the ability to manipulate other very long strings of ones and
zeroes. Even those few who actually understood what a computer operating
system was were apt to think of it as a fantastically arcane engineering
prodigy, like a breeder reactor or a U2 spy plane, and not something
that could ever be (in the parlance of high-tech) ``productized.''

Yet now the company that Gates and Allen founded is selling operating
systems like Gillette sells razor blades. New releases of operating
systems are launched as if they were Hollywood blockbusters, with
celebrity endorsements, talk show appearances, and world tours. The
market for them is vast enough that people worry about whether it has
been monopolized by one company. Even the least technically-minded
people in our society now have at least a hazy idea of what operating
systems do; what is more, they have strong opinions about their relative
merits. It is commonly understood, even by technically unsophisticated
computer users, that if you have a piece of software that works on your
Macintosh, and you move it over onto a Windows machine, it will not run.
That this would, in fact, be a laughable and idiotic mistake, like
nailing horseshoes to the tires of a Buick.

A person who went into a coma before Microsoft was founded, and woke up
now, could pick up this morning's New York Times and understand
everything in it --- almost:

Item: the richest man in the world made his fortune from--- what?
Railways? Shipping? Oil? No, operating systems. Item: the Department of
Justice is tackling Microsoft's supposed OS monopoly with legal tools
that were invented to restrain the power of Nineteenth-Century robber
barons. Item: a woman friend of mine recently told me that she'd broken
off a (hitherto) stimulating exchange of e-mail with a young man. At
first he had seemed like such an intelligent and interesting guy, she
said, but then ``he started going all PC-versus-Mac on me.''

What the hell is going on here? And does the operating system business
have a future, or only a past? Here is my view, which is entirely
subjective; but since I have spent a fair amount of time not only using,
but programming, Macintoshes, Windows machines, Linux boxes and the
BeOS, perhaps it is not so ill-informed as to be completely worthless.
This is a subjective essay, more review than research paper, and so it
might seem unfair or biased compared to the technical reviews you can
find in PC magazines. But ever since the Mac came out, our operating
systems have been based on metaphors, and anything with metaphors in it
is fair game as far as I'm concerned.

\section{MGBs, TANKS, AND BATMOBILES}

Around the time that Jobs, Wozniak, Gates, and Allen were dreaming up
these unlikely schemes, I was a teenager living in Ames, Iowa. One of my
friends' dads had an old MGB sports car rusting away in his garage.
Sometimes he would actually manage to get it running and then he would
take us for a spin around the block, with a memorable look of wild
youthful exhiliration on his face; to his worried passengers, he was a
madman, stalling and backfiring around Ames, Iowa and eating the dust of
rusty Gremlins and Pintos, but in his own mind he was Dustin Hoffman
tooling across the Bay Bridge with the wind in his hair.

In retrospect, this was telling me two things about people's
relationship to technology. One was that romance and image go a long way
towards shaping their opinions. If you doubt it (and if you have a lot
of spare time on your hands) just ask anyone who owns a Macintosh and
who, on those grounds, imagines him- or herself to be a member of an
oppressed minority group.

The other, somewhat subtler point, was that interface is very important.
Sure, the MGB was a lousy car in almost every way that counted: balky,
unreliable, underpowered. But it was fun to drive. It was responsive.
Every pebble on the road was felt in the bones, every nuance in the
pavement transmitted instantly to the driver's hands. He could listen to
the engine and tell what was wrong with it. The steering responded
immediately to commands from his hands. To us passengers it was a
pointless exercise in going nowhere---about as interesting as peering
over someone's shoulder while he punches numbers into a spreadsheet. But
to the driver it was an experience. For a short time he was extending
his body and his senses into a larger realm, and doing things that he
couldn't do unassisted.

The analogy between cars and operating systems is not half bad, and so
let me run with it for a moment, as a way of giving an executive summary
of our situation today.

Imagine a crossroads where four competing auto dealerships are situated.
One of them (Microsoft) is much, much bigger than the others. It started
out years ago selling three-speed bicycles (MS-DOS); these were not
perfect, but they worked, and when they broke you could easily fix them.

There was a competing bicycle dealership next door (Apple) that one day
began selling motorized vehicles---expensive but attractively styled
cars with their innards hermetically sealed, so that how they worked was
something of a mystery.

The big dealership responded by rushing a moped upgrade kit (the
original Windows) onto the market. This was a Rube Goldberg contraption
that, when bolted onto a three-speed bicycle, enabled it to keep up,
just barely, with Apple-cars. The users had to wear goggles and were
always picking bugs out of their teeth while Apple owners sped along in
hermetically sealed comfort, sneering out the windows. But the
Micro-mopeds were cheap, and easy to fix compared with the Apple-cars,
and their market share waxed.

Eventually the big dealership came out with a full-fledged car: a
colossal station wagon (Windows 95). It had all the aesthetic appeal of
a Soviet worker housing block, it leaked oil and blew gaskets, and it
was an enormous success. A little later, they also came out with a
hulking off-road vehicle intended for industrial users (Windows NT)
which was no more beautiful than the station wagon, and only a little
more reliable.

Since then there has been a lot of noise and shouting, but little has
changed. The smaller dealership continues to sell sleek Euro-styled
sedans and to spend a lot of money on advertising campaigns. They have
had GOING OUT OF BUSINESS! signs taped up in their windows for so long
that they have gotten all yellow and curly. The big one keeps making
bigger and bigger station wagons and ORVs.

On the other side of the road are two competitors that have come along
more recently.

One of them (Be, Inc.) is selling fully operational Batmobiles (the
BeOS). They are more beautiful and stylish even than the Euro-sedans,
better designed, more technologically advanced, and at least as reliable
as anything else on the market---and yet cheaper than the others.

With one exception, that is: Linux, which is right next door, and which
is not a business at all. It's a bunch of RVs, yurts, tepees, and
geodesic domes set up in a field and organized by consensus. The people
who live there are making tanks. These are not old-fashioned, cast-iron
Soviet tanks; these are more like the M1 tanks of the U.S. Army, made of
space-age materials and jammed with sophisticated technology from one
end to the other. But they are better than Army tanks. They've been
modified in such a way that they never, ever break down, are light and
maneuverable enough to use on ordinary streets, and use no more fuel
than a subcompact car. These tanks are being cranked out, on the spot,
at a terrific pace, and a vast number of them are lined up along the
edge of the road with keys in the ignition. Anyone who wants can simply
climb into one and drive it away for free.

Customers come to this crossroads in throngs, day and night. Ninety
percent of them go straight to the biggest dealership and buy station
wagons or off-road vehicles. They do not even look at the other
dealerships.

Of the remaining ten percent, most go and buy a sleek Euro-sedan,
pausing only to turn up their noses at the philistines going to buy the
station wagons and ORVs. If they even notice the people on the opposite
side of the road, selling the cheaper, technically superior vehicles,
these customers deride them cranks and half-wits.

The Batmobile outlet sells a few vehicles to the occasional car nut who
wants a second vehicle to go with his station wagon, but seems to
accept, at least for now, that it's a fringe player.

The group giving away the free tanks only stays alive because it is
staffed by volunteers, who are lined up at the edge of the street with
bullhorns, trying to draw customers' attention to this incredible
situation. A typical conversation goes something like this:

Hacker with bullhorn: ``Save your money! Accept one of our free tanks!
It is invulnerable, and can drive across rocks and swamps at ninety
miles an hour while getting a hundred miles to the gallon!''

Prospective station wagon buyer: ``I know what you say is
true\ldots{}but\ldots{}er\ldots{}I don't know how to maintain a tank!''

Bullhorn: ``You don't know how to maintain a station wagon either!''

Buyer: ``But this dealership has mechanics on staff. If something goes
wrong with my station wagon, I can take a day off work, bring it here,
and pay them to work on it while I sit in the waiting room for hours,
listening to elevator music.''

Bullhorn: ``But if you accept one of our free tanks we will send
volunteers to your house to fix it for free while you sleep!''

Buyer: ``Stay away from my house, you freak!''

Bullhorn: ``But\ldots{}''

Buyer: ``Can't you see that everyone is buying station wagons?''

\section{BIT-FLINGER}

The connection between cars, and ways of interacting with computers,
wouldn't have occurred to me at the time I was being taken for rides in
that MGB. I had signed up to take a computer programming class at Ames
High School. After a few introductory lectures, we students were granted
admission into a tiny room containing a teletype, a telephone, and an
old-fashioned modem consisting of a metal box with a pair of rubber cups
on the top (note: many readers, making their way through that last
sentence, probably felt an initial pang of dread that this essay was
about to turn into a tedious, codgerly reminiscence about how tough we
had it back in the old days; rest assured that I am actually positioning
my pieces on the chessboard, as it were, in preparation to make a point
about truly hip and up-to-the minute topics like Open Source Software).
The teletype was exactly the same sort of machine that had been used,
for decades, to send and receive telegrams. It was basically a loud
typewriter that could only produce UPPERCASE LETTERS. Mounted to one
side of it was a smaller machine with a long reel of paper tape on it,
and a clear plastic hopper underneath.

In order to connect this device (which was not a computer at all) to the
Iowa State University mainframe across town, you would pick up the
phone, dial the computer's number, listen for strange noises, and then
slam the handset down into the rubber cups. If your aim was true, one
would wrap its neoprene lips around the earpiece and the other around
the mouthpiece, consummating a kind of informational soixante-neuf. The
teletype would shudder as it was possessed by the spirit of the distant
mainframe, and begin to hammer out cryptic messages.

Since computer time was a scarce resource, we used a sort of batch
processing technique. Before dialing the phone, we would turn on the
tape puncher (a subsidiary machine bolted to the side of the teletype)
and type in our programs. Each time we depressed a key, the teletype
would bash out a letter on the paper in front of us, so we could read
what we'd typed; but at the same time it would convert the letter into a
set of eight binary digits, or bits, and punch a corresponding pattern
of holes across the width of a paper tape. The tiny disks of paper
knocked out of the tape would flutter down into the clear plastic
hopper, which would slowly fill up what can only be described as actual
bits. On the last day of the school year, the smartest kid in the class
(not me) jumped out from behind his desk and flung several quarts of
these bits over the head of our teacher, like confetti, as a sort of
semi-affectionate practical joke. The image of this man sitting there,
gripped in the opening stages of an atavistic fight-or-flight reaction,
with millions of bits (megabytes) sifting down out of his hair and into
his nostrils and mouth, his face gradually turning purple as he built up
to an explosion, is the single most memorable scene from my formal
education.

Anyway, it will have been obvious that my interaction with the computer
was of an extremely formal nature, being sharply divided up into
different phases, viz.: (1) sitting at home with paper and pencil, miles
and miles from any computer, I would think very, very hard about what I
wanted the computer to do, and translate my intentions into a computer
language---a series of alphanumeric symbols on a page. (2) I would carry
this across a sort of informational cordon sanitaire (three miles of
snowdrifts) to school and type those letters into a machine---not a
computer---which would convert the symbols into binary numbers and
record them visibly on a tape. (3) Then, through the rubber-cup modem, I
would cause those numbers to be sent to the university mainframe, which
would (4) do arithmetic on them and send different numbers back to the
teletype. (5) The teletype would convert these numbers back into letters
and hammer them out on a page and (6) I, watching, would construe the
letters as meaningful symbols.

The division of responsibilities implied by all of this is admirably
clean: computers do arithmetic on bits of information. Humans construe
the bits as meaningful symbols. But this distinction is now being
blurred, or at least complicated, by the advent of modern operating
systems that use, and frequently abuse, the power of metaphor to make
computers accessible to a larger audience. Along the way---possibly
because of those metaphors, which make an operating system a sort of
work of art---people start to get emotional, and grow attached to pieces
of software in the way that my friend's dad did to his MGB.

People who have only interacted with computers through graphical user
interfaces like the MacOS or Windows---which is to say, almost everyone
who has ever used a computer---may have been startled, or at least
bemused, to hear about the telegraph machine that I used to communicate
with a computer in 1973. But there was, and is, a good reason for using
this particular kind of technology. Human beings have various ways of
communicating to each other, such as music, art, dance, and facial
expressions, but some of these are more amenable than others to being
expressed as strings of symbols. Written language is the easiest of all,
because, of course, it consists of strings of symbols to begin with. If
the symbols happen to belong to a phonetic alphabet (as opposed to, say,
ideograms), converting them into bits is a trivial procedure, and one
that was nailed, technologically, in the early nineteenth century, with
the introduction of Morse code and other forms of telegraphy.

We had a human/computer interface a hundred years before we had
computers. When computers came into being around the time of the Second
World War, humans, quite naturally, communicated with them by simply
grafting them on to the already-existing technologies for translating
letters into bits and vice versa: teletypes and punch card machines.

These embodied two fundamentally different approaches to computing. When
you were using cards, you'd punch a whole stack of them and run them
through the reader all at once, which was called batch processing. You
could also do batch processing with a teletype, as I have already
described, by using the paper tape reader, and we were certainly
encouraged to use this approach when I was in high school. But---though
efforts were made to keep us unaware of this---the teletype could do
something that the card reader could not. On the teletype, once the
modem link was established, you could just type in a line and hit the
return key. The teletype would send that line to the computer, which
might or might not respond with some lines of its own, which the
teletype would hammer out---producing, over time, a transcript of your
exchange with the machine. This way of doing it did not even have a name
at the time, but when, much later, an alternative became available, it
was retroactively dubbed the Command Line Interface.

When I moved on to college, I did my computing in large, stifling rooms
where scores of students would sit in front of slightly updated versions
of the same machines and write computer programs: these used dot-matrix
printing mechanisms, but were (from the computer's point of view)
identical to the old teletypes. By that point, computers were better at
time-sharing---that is, mainframes were still mainframes, but they were
better at communicating with a large number of terminals at once.
Consequently, it was no longer necessary to use batch processing. Card
readers were shoved out into hallways and boiler rooms, and batch
processing became a nerds-only kind of thing, and consequently took on a
certain eldritch flavor among those of us who even knew it existed. We
were all off the Batch, and on the Command Line, interface now---my very
first shift in operating system paradigms, if only I'd known it.

A huge stack of accordion-fold paper sat on the floor underneath each
one of these glorified teletypes, and miles of paper shuddered through
their platens. Almost all of this paper was thrown away or recycled
without ever having been touched by ink---an ecological atrocity so
glaring that those machines soon replaced by video terminals---so-called
``glass teletypes''---which were quieter and didn't waste paper. Again,
though, from the computer's point of view these were indistinguishable
from World War II-era teletype machines. In effect we still used
Victorian technology to communicate with computers until about 1984,
when the Macintosh was introduced with its Graphical User Interface.
Even after that, the Command Line continued to exist as an underlying
stratum---a sort of brainstem reflex---of many modern computer systems
all through the heyday of Graphical User Interfaces, or GUIs as I will
call them from now on.

\section{GUIs}

Now the first job that any coder needs to do when writing a new piece of
software is to figure out how to take the information that is being
worked with (in a graphics program, an image; in a spreadsheet, a grid
of numbers) and turn it into a linear string of bytes. These strings of
bytes are commonly called files or (somewhat more hiply) streams. They
are to telegrams what modern humans are to Cro-Magnon man, which is to
say the same thing under a different name. All that you see on your
computer screen---your Tomb Raider, your digitized voice mail messages,
faxes, and word processing documents written in thirty-seven different
typefaces---is still, from the computer's point of view, just like
telegrams, except much longer, and demanding of more arithmetic.

The quickest way to get a taste of this is to fire up your web browser,
visit a site, and then select the View/Document Source menu item. You
will get a bunch of computer code that looks something like this:

\begin{lstlisting}
<HTML>
<HEAD>
    <TITLE> C R Y P T O N O M I C O N</TITLE>

</HEAD>
<BODY BGCOLOR="#000000" LINK="#996600" ALINK="#FFFFFF" VLINK="#663300">

<MAP NAME="navtext">
    <AREA SHAPE=RECT HREF="praise.html" COORDS="0,37,84,55">
    <AREA SHAPE=RECT HREF="author.html" COORDS="0,59,137,75">
    <AREA SHAPE=RECT HREF="text.html" COORDS="0,81,101,96">
    <AREA SHAPE=RECT HREF="tour.html" COORDS="0,100,121,117">
    <AREA SHAPE=RECT HREF="order.html" COORDS="0,122,143,138">
    <AREA SHAPE=RECT HREF="beginning.html" COORDS="0,140,213,157">
</MAP>


<CENTER>
<TABLE BORDER="0" CELLPADDING="0" CELLSPACING="0" WIDTH="520">
<TR>

    <TD VALIGN=TOP ROWSPAN="5">
    <IMG SRC="images/spacer.gif" WIDTH="30" HEIGHT="1" BORDER="0">
    </TD>

    <TD VALIGN=TOP COLSPAN="2">
    <IMG SRC="images/main_banner.gif" ALT="Cryptonomincon by Neal
Stephenson" WIDTH="479" HEIGHT="122" BORDER="0">
    </TD>

</TR>  
\end{lstlisting}
This crud is called HTML (HyperText Markup Language) and it is basically
a very simple programming language instructing your web browser how to
draw a page on a screen. Anyone can learn HTML and many people do. The
important thing is that no matter what splendid multimedia web pages
they might represent, HTML files are just telegrams.

When Ronald Reagan was a radio announcer, he used to call baseball games
by reading the terse descriptions that trickled in over the telegraph
wire and were printed out on a paper tape. He would sit there, all by
himself in a padded room with a microphone, and the paper tape would eke
out of the machine and crawl over the palm of his hand printed with
cryptic abbreviations. If the count went to three and two, Reagan would
describe the scene as he saw it in his mind's eye: ``The brawny
left-hander steps out of the batter's box to wipe the sweat from his
brow. The umpire steps forward to sweep the dirt from home plate.'' and
so on. When the cryptogram on the paper tape announced a base hit, he
would whack the edge of the table with a pencil, creating a little sound
effect, and describe the arc of the ball as if he could actually see it.
His listeners, many of whom presumably thought that Reagan was actually
at the ballpark watching the game, would reconstruct the scene in their
minds according to his descriptions.

This is exactly how the World Wide Web works: the HTML files are the
pithy description on the paper tape, and your Web browser is Ronald
Reagan. The same is true of Graphical User Interfaces in general.

So an OS is a stack of metaphors and abstractions that stands between
you and the telegrams, and embodying various tricks the programmer used
to convert the information you're working with---be it images, e-mail
messages, movies, or word processing documents---into the necklaces of
bytes that are the only things computers know how to work with. When we
used actual telegraph equipment (teletypes) or their higher-tech
substitutes (``glass teletypes,'' or the MS-DOS command line) to work
with our computers, we were very close to the bottom of that stack. When
we use most modern operating systems, though, our interaction with the
machine is heavily mediated. Everything we do is interpreted and
translated time and again as it works its way down through all of the
metaphors and abstractions.

The Macintosh OS was a revolution in both the good and bad senses of
that word. Obviously it was true that command line interfaces were not
for everyone, and that it would be a good thing to make computers more
accessible to a less technical audience---if not for altruistic reasons,
then because those sorts of people constituted an incomparably vaster
market. It was clear the the Mac's engineers saw a whole new country
stretching out before them; you could almost hear them muttering, ``Wow!
We don't have to be bound by files as linear streams of bytes anymore,
vive la revolution, let's see how far we can take this!'' No command
line interface was available on the Macintosh; you talked to it with the
mouse, or not at all. This was a statement of sorts, a credential of
revolutionary purity. It seemed that the designers of the Mac intended
to sweep Command Line Interfaces into the dustbin of history.

My own personal love affair with the Macintosh began in the spring of
1984 in a computer store in Cedar Rapids, Iowa, when a friend of
mine---coincidentally, the son of the MGB owner---showed me a Macintosh
running MacPaint, the revolutionary drawing program. It ended in July of
1995 when I tried to save a big important file on my Macintosh Powerbook
and instead instead of doing so, it annihilated the data so thoroughly
that two different disk crash utility programs were unable to find any
trace that it had ever existed. During the intervening ten years, I had
a passion for the MacOS that seemed righteous and reasonable at the time
but in retrospect strikes me as being exactly the same sort of goofy
infatuation that my friend's dad had with his car.

The introduction of the Mac triggered a sort of holy war in the computer
world. Were GUIs a brilliant design innovation that made computers more
human-centered and therefore accessible to the masses, leading us toward
an unprecedented revolution in human society, or an insulting bit of
audiovisual gimcrackery dreamed up by flaky Bay Area hacker types that
stripped computers of their power and flexibility and turned the noble
and serious work of computing into a childish video game?

This debate actually seems more interesting to me today than it did in
the mid--1980s. But people more or less stopped debating it when
Microsoft endorsed the idea of GUIs by coming out with the first
Windows. At this point, command-line partisans were relegated to the
status of silly old grouches, and a new conflict was touched off,
between users of MacOS and users of Windows.

There was plenty to argue about. The first Macintoshes looked different
from other PCs even when they were turned off: they consisted of one box
containing both CPU (the part of the computer that does arithmetic on
bits) and monitor screen. This was billed, at the time, as a
philosophical statement of sorts: Apple wanted to make the personal
computer into an appliance, like a toaster. But it also reflected the
purely technical demands of running a graphical user interface. In a GUI
machine, the chips that draw things on the screen have to be integrated
with the computer's central processing unit, or CPU, to a far greater
extent than is the case with command-line interfaces, which until
recently didn't even know that they weren't just talking to teletypes.

This distinction was of a technical and abstract nature, but it became
clearer when the machine crashed (it is commonly the case with
technologies that you can get the best insight about how they work by
watching them fail). When everything went to hell and the CPU began
spewing out random bits, the result, on a CLI machine, was lines and
lines of perfectly formed but random characters on the screen---known to
cognoscenti as ``going Cyrillic.'' But to the MacOS, the screen was not
a teletype, but a place to put graphics; the image on the screen was a
bitmap, a literal rendering of the contents of a particular portion of
the computer's memory. When the computer crashed and wrote gibberish
into the bitmap, the result was something that looked vaguely like
static on a broken television set---a ``snow crash.''

And even after the introduction of Windows, the underlying differences
endured; when a Windows machine got into trouble, the old command-line
interface would fall down over the GUI like an asbestos fire curtain
sealing off the proscenium of a burning opera. When a Macintosh got into
trouble it presented you with a cartoon of a bomb, which was funny the
first time you saw it.

And these were by no means superficial differences. The reversion of
Windows to a CLI when it was in distress proved to Mac partisans that
Windows was nothing more than a cheap facade, like a garish afghan flung
over a rotted-out sofa. They were disturbed and annoyed by the sense
that lurking underneath Windows' ostensibly user-friendly interface
was---literally---a subtext.

For their part, Windows fans might have made the sour observation that
all computers, even Macintoshes, were built on that same subtext, and
that the refusal of Mac owners to admit that fact to themselves seemed
to signal a willingness, almost an eagerness, to be duped.

Anyway, a Macintosh had to switch individual bits in the memory chips on
the video card, and it had to do it very fast, and in arbitrarily
complicated patterns. Nowadays this is cheap and easy, but in the
technological regime that prevailed in the early 1980s, the only
realistic way to do it was to build the motherboard (which contained the
CPU) and the video system (which contained the memory that was mapped
onto the screen) as a tightly integrated whole---hence the single,
hermetically sealed case that made the Macintosh so distinctive.

When Windows came out, it was conspicuous for its ugliness, and its
current successors, Windows 95 and Windows NT, are not things that
people would pay money to look at either. Microsoft's complete disregard
for aesthetics gave all of us Mac-lovers plenty of opportunities to look
down our noses at them. That Windows looked an awful lot like a direct
ripoff of MacOS gave us a burning sense of moral outrage to go with it.
Among people who really knew and appreciated computers (hackers, in
Steven Levy's non-pejorative sense of that word) and in a few other
niches such as professional musicians, graphic artists and
schoolteachers, the Macintosh, for a while, was simply the computer. It
was seen as not only a superb piece of engineering, but an embodiment of
certain ideals about the use of technology to benefit mankind, while
Windows was seen as a pathetically clumsy imitation and a sinister world
domination plot rolled into one. So very early, a pattern had been
established that endures to this day: people dislike Microsoft, which is
okay; but they dislike it for reasons that are poorly considered, and in
the end, self-defeating.

\section{CLASS STRUGGLE ON THE DESKTOP}

Now that the Third Rail has been firmly grasped, it is worth reviewing
some basic facts here: like any other publicly traded, for-profit
corporation, Microsoft has, in effect, borrowed a bunch of money from
some people (its stockholders) in order to be in the bit business. As an
officer of that corporation, Bill Gates has one responsibility only,
which is to maximize return on investment. He has done this incredibly
well. Any actions taken in the world by Microsoft-any software released
by them, for example---are basically epiphenomena, which can't be
interpreted or understood except insofar as they reflect Bill Gates's
execution of his one and only responsibility.

It follows that if Microsoft sells goods that are aesthetically
unappealing, or that don't work very well, it does not mean that they
are (respectively) philistines or half-wits. It is because Microsoft's
excellent management has figured out that they can make more money for
their stockholders by releasing stuff with obvious, known imperfections
than they can by making it beautiful or bug-free. This is annoying, but
(in the end) not half so annoying as watching Apple inscrutably and
relentlessly destroy itself.

Hostility towards Microsoft is not difficult to find on the Net, and it
blends two strains: resentful people who feel Microsoft is too powerful,
and disdainful people who think it's tacky. This is all strongly
reminiscent of the heyday of Communism and Socialism, when the
bourgeoisie were hated from both ends: by the proles, because they had
all the money, and by the intelligentsia, because of their tendency to
spend it on lawn ornaments. Microsoft is the very embodiment of modern
high-tech prosperity---it is, in a word, bourgeois---and so it attracts
all of the same gripes.

The opening ``splash screen'' for Microsoft Word 6.0 summed it up pretty
neatly: when you started up the program you were treated to a picture of
an expensive enamel pen lying across a couple of sheets of fancy-looking
handmade writing paper. It was obviously a bid to make the software look
classy, and it might have worked for some, but it failed for me, because
the pen was a ballpoint, and I'm a fountain pen man. If Apple had done
it, they would've used a Mont Blanc fountain pen, or maybe a Chinese
calligraphy brush. And I doubt that this was an accident. Recently I
spent a while re-installing Windows NT on one of my home computers, and
many times had to double-click on the ``Control Panel'' icon. For
reasons that are difficult to fathom, this icon consists of a picture of
a clawhammer and a chisel or screwdriver resting on top of a file
folder.

These aesthetic gaffes give one an almost uncontrollable urge to make
fun of Microsoft, but again, it is all beside the point---if Microsoft
had done focus group testing of possible alternative graphics, they
probably would have found that the average mid-level office worker
associated fountain pens with effete upper management toffs and was more
comfortable with ballpoints. Likewise, the regular guys, the balding
dads of the world who probably bear the brunt of setting up and
maintaining home computers, can probably relate better to a picture of a
clawhammer---while perhaps harboring fantasies of taking a real one to
their balky computers.

This is the only way I can explain certain peculiar facts about the
current market for operating systems, such as that ninety percent of all
customers continue to buy station wagons off the Microsoft lot while
free tanks are there for the taking, right across the street.

A string of ones and zeroes was not a difficult thing for Bill Gates to
distribute, one he'd thought of the idea. The hard part was selling
it---reassuring customers that they were actually getting something in
return for their money.

Anyone who has ever bought a piece of software in a store has had the
curiously deflating experience of taking the bright shrink-wrapped box
home, tearing it open, finding that it's 95 percent air, throwing away
all the little cards, party favors, and bits of trash, and loading the
disk into the computer. The end result (after you've lost the disk) is
nothing except some images on a computer screen, and some capabilities
that weren't there before. Sometimes you don't even have that---you have
a string of error messages instead. But your money is definitely gone.
Now we are almost accustomed to this, but twenty years ago it was a very
dicey business proposition. Bill Gates made it work anyway. He didn't
make it work by selling the best software or offering the cheapest
price. Instead he somehow got people to believe that they were receiving
something in exchange for their money.

The streets of every city in the world are filled with those hulking,
rattling station wagons. Anyone who doesn't own one feels a little
weird, and wonders, in spite of himself, whether it might not be time to
cease resistance and buy one; anyone who does, feels confident that he
has acquired some meaningful possession, even on those days when the
vehicle is up on a lift in an auto repair shop.

All of this is perfectly congruent with membership in the bourgeoisie,
which is as much a mental, as a material state. And it explains why
Microsoft is regularly attacked, on the Net, from both sides. People who
are inclined to feel poor and oppressed construe everything Microsoft
does as some sinister Orwellian plot. People who like to think of
themselves as intelligent and informed technology users are driven crazy
by the clunkiness of Windows.

Nothing is more annoying to sophisticated people to see someone who is
rich enough to know better being tacky---unless it is to realize, a
moment later, that they probably know they are tacky and they simply
don't care and they are going to go on being tacky, and rich, and happy,
forever. Microsoft therefore bears the same relationship to the Silicon
Valley elite as the Beverly Hillbillies did to their fussy banker,
Mr.~Drysdale---who is irritated not so much by the fact that the
Clampetts moved to his neighborhood as by the knowledge that, when
Jethro is seventy years old, he's still going to be talking like a
hillbilly and wearing bib overalls, and he's still going to be a lot
richer than Mr.~Drysdale.

Even the hardware that Windows ran on, when compared to the machines put
out by Apple, looked like white-trash stuff, and still mostly does. The
reason was that Apple was and is a hardware company, while Microsoft was
and is a software company. Apple therefore had a monopoly on hardware
that could run MacOS, whereas Windows-compatible hardware came out of a
free market. The free market seems to have decided that people will not
pay for cool-looking computers; PC hardware makers who hire designers to
make their stuff look distinctive get their clocks cleaned by Taiwanese
clone makers punching out boxes that look as if they belong on
cinderblocks in front of someone's trailer. But Apple could make their
hardware as pretty as they wanted to and simply pass the higher prices
on to their besotted consumers, like me. Only last week (I am writing
this sentence in early Jan. 1999) the technology sections of all the
newspapers were filled with adulatory press coverage of how Apple had
released the iMac in several happenin' new colors like Blueberry and
Tangerine.

Apple has always insisted on having a hardware monopoly, except for a
brief period in the mid--1990s when they allowed clone-makers to compete
with them, before subsequently putting them out of business. Macintosh
hardware was, consequently, expensive. You didn't open it up and fool
around with it because doing so would void the warranty. In fact the
first Mac was specifically designed to be difficult to open---you needed
a kit of exotic tools, which you could buy through little ads that began
to appear in the back pages of magazines a few months after the Mac came
out on the market. These ads always had a certain disreputable air about
them, like pitches for lock-picking tools in the backs of lurid
detective magazines.

This monopolistic policy can be explained in at least three different
ways.

\begin{description}
\item[THE CHARITABLE EXPLANATION]
is that the hardware monopoly policy reflected a drive on Apple's part
to provide a seamless, unified blending of hardware, operating system,
and software. There is something to this. It is hard enough to make an
OS that works well on one specific piece of hardware, designed and
tested by engineers who work down the hallway from you, in the same
company. Making an OS to work on arbitrary pieces of hardware, cranked
out by rabidly entrepeneurial clonemakers on the other side of the
International Date Line, is very difficult, and accounts for much of the
troubles people have using Windows.

\item[THE FINANCIAL EXPLANATION]
is that Apple, unlike Microsoft, is and always has been a hardware
company. It simply depends on revenue from selling hardware, and cannot
exist without it.

\item[THE NOT-SO-CHARITABLE EXPLANATION]
has to do with Apple's corporate culture, which is rooted in Bay Area
Baby Boomdom.

\end{description}
Now, since I'm going to talk for a moment about culture, full disclosure
is probably in order, to protect myself against allegations of conflict
of interest and ethical turpitude: (1) Geographically I am a Seattleite,
of a Saturnine temperament, and inclined to take a sour view of the
Dionysian Bay Area, just as they tend to be annoyed and appalled by us.
(2) Chronologically I am a post-Baby Boomer. I feel that way, at least,
because I never experienced the fun and exciting parts of the whole
Boomer scene---just spent a lot of time dutifully chuckling at Boomers'
maddeningly pointless anecdotes about just how stoned they got on
various occasions, and politely fielding their assertions about how
great their music was. But even from this remove it was possible to
glean certain patterns, and one that recurred as regularly as an urban
legend was the one about how someone would move into a commune populated
by sandal-wearing, peace-sign flashing flower children, and eventually
discover that, underneath this facade, the guys who ran it were actually
control freaks; and that, as living in a commune, where much lip service
was paid to ideals of peace, love and harmony, had deprived them of
normal, socially approved outlets for their control-freakdom, it tended
to come out in other, invariably more sinister, ways.

Applying this to the case of Apple Computer will be left as an exercise
for the reader, and not a very difficult exercise.

It is a bit unsettling, at first, to think of Apple as a control freak,
because it is completely at odds with their corporate image. Weren't
these the guys who aired the famous Super Bowl ads showing suited,
blindfolded executives marching like lemmings off a cliff? Isn't this
the company that even now runs ads picturing the Dalai Lama (except in
Hong Kong) and Einstein and other offbeat rebels?

It is indeed the same company, and the fact that they have been able to
plant this image of themselves as creative and rebellious free-thinkers
in the minds of so many intelligent and media-hardened skeptics really
gives one pause. It is testimony to the insidious power of expensive
slick ad campaigns and, perhaps, to a certain amount of wishful thinking
in the minds of people who fall for them. It also raises the question of
why Microsoft is so bad at PR, when the history of Apple demonstrates
that, by writing large checks to good ad agencies, you can plant a
corporate image in the minds of intelligent people that is completely at
odds with reality. (The answer, for people who don't like Damoclean
questions, is that since Microsoft has won the hearts and minds of the
silent majority---the bourgeoisie---they don't give a damn about having
a slick image, any more then Dick Nixon did. ``I want to
believe,''---the mantra that Fox Mulder has pinned to his office wall in
The X-Files---applies in different ways to these two companies; Mac
partisans want to believe in the image of Apple purveyed in those ads,
and in the notion that Macs are somehow fundamentally different from
other computers, while Windows people want to believe that they are
getting something for their money, engaging in a respectable business
transaction).

In any event, as of 1987, both MacOS and Windows were out on the market,
running on hardware platforms that were radically different from each
other---not only in the sense that MacOS used Motorola CPU chips while
Windows used Intel, but in the sense---then overlooked, but in the long
run, vastly more significant---that the Apple hardware business was a
rigid monopoly and the Windows side was a churning free-for-all.

But the full ramifications of this did not become clear until very
recently---in fact, they are still unfolding, in remarkably strange
ways, as I'll explain when we get to Linux. The upshot is that millions
of people got accustomed to using GUIs in one form or another. By doing
so, they made Apple/Microsoft a lot of money. The fortunes of many
people have become bound up with the ability of these companies to
continue selling products whose salability is very much open to
question.

\section{HONEY-POT, TAR-PIT, WHATEVER}

When Gates and Allen invented the idea of selling software, they ran
into criticism from both hackers and sober-sided businesspeople. Hackers
understood that software was just information, and objected to the idea
of selling it. These objections were partly moral. The hackers were
coming out of the scientific and academic world where it is imperative
to make the results of one's work freely available to the public. They
were also partly practical; how can you sell something that can be
easily copied? Businesspeople, who are polar opposites of hackers in so
many ways, had objections of their own. Accustomed to selling toasters
and insurance policies, they naturally had a difficult time
understanding how a long collection of ones and zeroes could constitute
a salable product.

Obviously Microsoft prevailed over these objections, and so did Apple.
But the objections still exist. The most hackerish of all the hackers,
the Ur-hacker as it were, was and is Richard Stallman, who became so
annoyed with the evil practice of selling software that, in 1984 (the
same year that the Macintosh went on sale) he went off and founded
something called the Free Software Foundation, which commenced work on
something called GNU. Gnu is an acronym for Gnu's Not Unix, but this is
a joke in more ways than one, because GNU most certainly IS Unix,.
Because of trademark concerns (``Unix'' is trademarked by AT\&T) they
simply could not claim that it was Unix, and so, just to be extra safe,
they claimed that it wasn't. Notwithstanding the incomparable talent and
drive possessed by Mr.~Stallman and other GNU adherents, their project
to build a free Unix to compete against Microsoft and Apple's OSes was a
little bit like trying to dig a subway system with a teaspoon. Until,
that is, the advent of Linux, which I will get to later.

But the basic idea of re-creating an operating system from scratch was
perfectly sound and completely doable. It has been done many times. It
is inherent in the very nature of operating systems.

Operating systems are not strictly necessary. There is no reason why a
sufficiently dedicated coder could not start from nothing with every
project and write fresh code to handle such basic, low-level operations
as controlling the read/write heads on the disk drives and lighting up
pixels on the screen. The very first computers had to be programmed in
this way. But since nearly every program needs to carry out those same
basic operations, this approach would lead to vast duplication of
effort.

Nothing is more disagreeable to the hacker than duplication of effort.
The first and most important mental habit that people develop when they
learn how to write computer programs is to generalize, generalize,
generalize. To make their code as modular and flexible as possible,
breaking large problems down into small subroutines that can be used
over and over again in different contexts. Consequently, the development
of operating systems, despite being technically unnecessary, was
inevitable. Because at its heart, an operating system is nothing more
than a library containing the most commonly used code, written once (and
hopefully written well) and then made available to every coder who needs
it.

So a proprietary, closed, secret operating system is a contradiction in
terms. It goes against the whole point of having an operating system.
And it is impossible to keep them secret anyway. The source code---the
original lines of text written by the programmers---can be kept secret.
But an OS as a whole is a collection of small subroutines that do very
specific, very clearly defined jobs. Exactly what those subroutines do
has to be made public, quite explicitly and exactly, or else the OS is
completely useless to programmers; they can't make use of those
subroutines if they don't have a complete and perfect understanding of
what the subroutines do.

The only thing that isn't made public is exactly how the subroutines do
what they do. But once you know what a subroutine does, it's generally
quite easy (if you are a hacker) to write one of your own that does
exactly the same thing. It might take a while, and it is tedious and
unrewarding, but in most cases it's not really hard.

What's hard, in hacking as in fiction, is not writing; it's deciding
what to write. And the vendors of commercial OSes have already decided,
and published their decisions.

This has been generally understood for a long time. MS-DOS was
duplicated, functionally, by a rival product, written from scratch,
called ProDOS, that did all of the same things in pretty much the same
way. In other words, another company was able to write code that did all
of the same things as MS-DOS and sell it at a profit. If you are using
the Linux OS, you can get a free program called WINE which is a windows
emulator; that is, you can open up a window on your desktop that runs
windows programs. It means that a completely functional Windows OS has
been recreated inside of Unix, like a ship in a bottle. And Unix itself,
which is vastly more sophisticated than MS-DOS, has been built up from
scratch many times over. Versions of it are sold by Sun,
Hewlett-Packard, AT\&T, Silicon Graphics, IBM, and others.

People have, in other words, been re-writing basic OS code for so long
that all of the technology that constituted an ``operating system'' in
the traditional (pre-GUI) sense of that phrase is now so cheap and
common that it's literally free. Not only could Gates and Allen not sell
MS-DOS today, they could not even give it away, because much more
powerful OSes are already being given away. Even the original Windows
(which was the only windows until 1995) has become worthless, in that
there is no point in owning something that can be emulated inside of
Linux---which is, itself, free.

In this way the OS business is very different from, say, the car
business. Even an old rundown car has some value. You can use it for
making runs to the dump, or strip it for parts. It is the fate of
manufactured goods to slowly and gently depreciate as they get old and
have to compete against more modern products.

But it is the fate of operating systems to become free.

Microsoft is a great software applications company. Applications---such
as Microsoft Word---are an area where innovation brings real, direct,
tangible benefits to users. The innovations might be new technology
straight from the research department, or they might be in the category
of bells and whistles, but in any event they are frequently useful and
they seem to make users happy. And Microsoft is in the process of
becoming a great research company. But Microsoft is not such a great
operating systems company. And this is not necessarily because their
operating systems are all that bad from a purely technological
standpoint. Microsoft's OSes do have their problems, sure, but they are
vastly better than they used to be, and they are adequate for most
people.

Why, then, do I say that Microsoft is not such a great operating systems
company? Because the very nature of operating systems is such that it is
senseless for them to be developed and owned by a specific company. It's
a thankless job to begin with. Applications create possibilities for
millions of credulous users, whereas OSes impose limitations on
thousands of grumpy coders, and so OS-makers will forever be on the
shit-list of anyone who counts for anything in the high-tech world.
Applications get used by people whose big problem is understanding all
of their features, whereas OSes get hacked by coders who are annoyed by
their limitations. The OS business has been good to Microsoft only
insofar as it has given them the money they needed to launch a really
good applications software business and to hire a lot of smart
researchers. Now it really ought to be jettisoned, like a spent booster
stage from a rocket. The big question is whether Microsoft is capable of
doing this. Or is it addicted to OS sales in the same way as Apple is to
selling hardware?

Keep in mind that Apple's ability to monopolize its own hardware supply
was once cited, by learned observers, as a great advantage over
Microsoft. At the time, it seemed to place them in a much stronger
position. In the end, it nearly killed them, and may kill them yet. The
problem, for Apple, was that most of the world's computer users ended up
owning cheaper hardware. But cheap hardware couldn't run MacOS, and so
these people switched to Windows.

Replace ``hardware'' with ``operating systems,'' and ``Apple'' with
``Microsoft'' and you can see the same thing about to happen all over
again. Microsoft dominates the OS market, which makes them money and
seems like a great idea for now. But cheaper and better OSes are
available, and they are growingly popular in parts of the world that are
not so saturated with computers as the US. Ten years from now, most of
the world's computer users may end up owning these cheaper OSes. But
these OSes do not, for the time being, run any Microsoft applications,
and so these people will use something else.

To put it more directly: every time someone decides to use a
non-Microsoft OS, Microsoft's OS division, obviously, loses a customer.
But, as things stand now, Microsoft's applications division loses a
customer too. This is not such a big deal as long as almost everyone
uses Microsoft OSes. But as soon as Windows' market share begins to
slip, the math starts to look pretty dismal for the people in Redmond.

This argument could be countered by saying that Microsoft could simply
re-compile its applications to run under other OSes. But this strategy
goes against most normal corporate instincts. Again the case of Apple is
instructive. When things started to go south for Apple, they should have
ported their OS to cheap PC hardware. But they didn't. Instead, they
tried to make the most of their brilliant hardware, adding new features
and expanding the product line. But this only had the effect of making
their OS more dependent on these special hardware features, which made
it worse for them in the end.

Likewise, when Microsoft's position in the OS world is threatened, their
corporate instincts will tell them to pile more new features into their
operating systems, and then re-jigger their software applications to
exploit those special features. But this will only have the effect of
making their applications dependent on an OS with declining market
share, and make it worse for them in the end.

The operating system market is a death-trap, a tar-pit, a slough of
despond. There are only two reasons to invest in Apple and Microsoft.
(1) each of these companies is in what we would call a co-dependency
relationship with their customers. The customers Want To Believe, and
Apple and Microsoft know how to give them what they want. (2) each
company works very hard to add new features to their OSes, which works
to secure customer loyalty, at least for a little while.

Accordingly, most of the remainder of this essay will be about those two
topics.

\section{THE TECHNOSPHERE}

Unix is the only OS remaining whose GUI (a vast suite of code called the
X Windows System) is separate from the OS in the old sense of the
phrase. This is to say that you can run Unix in pure command-line mode
if you want to, with no windows, icons, mouses, etc. whatsoever, and it
will still be Unix and capable of doing everything Unix is supposed to
do. But the other OSes: MacOS, the Windows family, and BeOS, have their
GUIs tangled up with the old-fashioned OS functions to the extent that
they have to run in GUI mode, or else they are not really running. So
it's no longer really possible to think of GUIs as being distinct from
the OS; they're now an inextricable part of the OSes that they belong
to---and they are by far the largest part, and by far the most expensive
and difficult part to create.

There are only two ways to sell a product: price and features. When OSes
are free, OS companies cannot compete on price, and so they compete on
features. This means that they are always trying to outdo each other
writing code that, until recently, was not considered to be part of an
OS at all: stuff like GUIs. This explains a lot about how these
companies behave.

It explains why Microsoft added a browser to their OS, for example. It
is easy to get free browsers, just as to get free OSes. If browsers are
free, and OSes are free, it would seem that there is no way to make
money from browsers or OSes. But if you can integrate a browser into the
OS and thereby imbue both of them with new features, you have a salable
product.

Setting aside, for the moment, the fact that this makes government
anti-trust lawyers really mad, this strategy makes sense. At least, it
makes sense if you assume (as Microsoft's management appears to) that
the OS has to be protected at all costs. The real question is whether
every new technological trend that comes down the pike ought to be used
as a crutch to maintain the OS's dominant position. Confronted with the
Web phenomenon, Microsoft had to develop a really good web browser, and
they did. But then they had a choice: they could have made that browser
work on many different OSes, which would give Microsoft a strong
position in the Internet world no matter what happened to their OS
market share. Or they could make the browser one with the OS, gambling
that this would make the OS look so modern and sexy that it would help
to preserve their dominance in that market. The problem is that when
Microsoft's OS position begins to erode (and since it is currently at
something like ninety percent, it can't go anywhere but down) it will
drag everything else down with it.

In your high school geology class you probably were taught that all life
on earth exists in a paper-thin shell called the biosphere, which is
trapped between thousands of miles of dead rock underfoot, and cold dead
radioactive empty space above. Companies that sell OSes exist in a sort
of technosphere. Underneath is technology that has already become free.
Above is technology that has yet to be developed, or that is too crazy
and speculative to be productized just yet. Like the Earth's biosphere,
the technosphere is very thin compared to what is above and what is
below.

But it moves a lot faster. In various parts of our world, it is possible
to go and visit rich fossil beds where skeleton lies piled upon
skeleton, recent ones on top and more ancient ones below. In theory they
go all the way back to the first single-celled organisms. And if you use
your imagination a bit, you can understand that, if you hang around long
enough, you'll become fossilized there too, and in time some more
advanced organism will become fossilized on top of you.

The fossil record---the La Brea Tar Pit---of software technology is the
Internet. Anything that shows up there is free for the taking (possibly
illegal, but free). Executives at companies like Microsoft must get used
to the experience---unthinkable in other industries---of throwing
millions of dollars into the development of new technologies, such as
Web browsers, and then seeing the same or equivalent software show up on
the Internet two years, or a year, or even just a few months, later.

By continuing to develop new technologies and add features onto their
products they can keep one step ahead of the fossilization process, but
on certain days they must feel like mammoths caught at La Brea, using
all their energies to pull their feet, over and over again, out of the
sucking hot tar that wants to cover and envelop them.

Survival in this biosphere demands sharp tusks and heavy, stomping feet
at one end of the organization, and Microsoft famously has those. But
trampling the other mammoths into the tar can only keep you alive for so
long. The danger is that in their obsession with staying out of the
fossil beds, these companies will forget about what lies above the
biosphere: the realm of new technology. In other words, they must hang
onto their primitive weapons and crude competitive instincts, but also
evolve powerful brains. This appears to be what Microsoft is doing with
its research division, which has been hiring smart people right and left
(Here I should mention that although I know, and socialize with, several
people in that company's research division, we never talk about business
issues and I have little to no idea what the hell they are up to. I have
learned much more about Microsoft by using the Linux operating system
than I ever would have done by using Windows).

Never mind how Microsoft used to make money; today, it is making its
money on a kind of temporal arbitrage. ``Arbitrage,'' in the usual
sense, means to make money by taking advantage of differences in the
price of something between different markets. It is spatial, in other
words, and hinges on the arbitrageur knowing what is going on
simultaneously in different places. Microsoft is making money by taking
advantage of differences in the price of technology in different times.
Temporal arbitrage, if I may coin a phrase, hinges on the arbitrageur
knowing what technologies people will pay money for next year, and how
soon afterwards those same technologies will become free. What spatial
and temporal arbitrage have in common is that both hinge on the
arbitrageur's being extremely well-informed; one about price gradients
across space at a given time, and the other about price gradients over
time in a given place.

So Apple/Microsoft shower new features upon their users almost daily, in
the hopes that a steady stream of genuine technical innovations,
combined with the ``I want to believe'' phenomenon, will prevent their
customers from looking across the road towards the cheaper and better
OSes that are available to them. The question is whether this makes
sense in the long run. If Microsoft is addicted to OSes as Apple is to
hardware, then they will bet the whole farm on their OSes, and tie all
of their new applications and technologies to them. Their continued
survival will then depend on these two things: adding more features to
their OSes so that customers will not switch to the cheaper
alternatives, and maintaining the image that, in some mysterious way,
gives those customers the feeling that they are getting something for
their money.

The latter is a truly strange and interesting cultural phenomenon.

\section{THE INTERFACE CULTURE}

A few years ago I walked into a grocery store somewhere and was
presented with the following tableau vivant: near the entrance a young
couple were standing in front of a large cosmetics display. The man was
stolidly holding a shopping basket between his hands while his mate
raked blister-packs of makeup off the display and piled them in. Since
then I've always thought of that man as the personification of an
interesting human tendency: not only are we not offended to be dazzled
by manufactured images, but we like it. We practically insist on it. We
are eager to be complicit in our own dazzlement: to pay money for a
theme park ride, vote for a guy who's obviously lying to us, or stand
there holding the basket as it's filled up with cosmetics.

I was in Disney World recently, specifically the part of it called the
Magic Kingdom, walking up Main Street USA. This is a perfect
gingerbready Victorian small town that culminates in a Disney castle. It
was very crowded; we shuffled rather than walked. Directly in front of
me was a man with a camcorder. It was one of the new breed of camcorders
where instead of peering through a viewfinder you gaze at a flat-panel
color screen about the size of a playing card, which televises live
coverage of whatever the camcorder is seeing. He was holding the
appliance close to his face, so that it obstructed his view. Rather than
go see a real small town for free, he had paid money to see a pretend
one, and rather than see it with the naked eye he was watching it on
television.

And rather than stay home and read a book, I was watching him.

Americans' preference for mediated experiences is obvious enough, and
I'm not going to keep pounding it into the ground. I'm not even going to
make snotty comments about it---after all, I was at Disney World as a
paying customer. But it clearly relates to the colossal success of GUIs
and so I have to talk about it some. Disney does mediated experiences
better than anyone. If they understood what OSes are, and why people use
them, they could crush Microsoft in a year or two.

In the part of Disney World called the Animal Kingdom there is a new
attraction, slated to open in March 1999, called the Maharajah Jungle
Trek. It was open for sneak previews when I was there. This is a
complete stone-by-stone reproduction of a hypothetical ruin in the
jungles of India. According to its backstory, it was built by a local
rajah in the 16th Century as a game reserve. He would go there with his
princely guests to hunt Bengal tigers. As time went on it fell into
disrepair and the tigers and monkeys took it over; eventually, around
the time of India's independence, it became a government wildlife
reserve, now open to visitors.

The place looks more like what I have just described than any actual
building you might find in India. All the stones in the broken walls are
weathered as if monsoon rains had been trickling down them for
centuries, the paint on the gorgeous murals is flaked and faded just so,
and Bengal tigers loll amid stumps of broken columns. Where modern
repairs have been made to the ancient structure, they've been done, not
as Disney's engineers would do them, but as thrifty Indian janitors
would---with hunks of bamboo and rust-spotted hunks of rebar. The rust
is painted on, or course, and protected from real rust by a plastic
clear-coat, but you can't tell unless you get down on your knees.

In one place you walk along a stone wall with a series of old pitted
friezes carved into it. One end of the wall has broken off and settled
into the earth, perhaps because of some long-forgotten earthquake, and
so a broad jagged crack runs across a panel or two, but the story is
still readable: first, primordial chaos leads to a flourishing of many
animal species. Next, we see the Tree of Life surrounded by diverse
animals. This is an obvious allusion (or, in showbiz lingo, a tie-in) to
the gigantic Tree of Life that dominates the center of Disney's Animal
Kingdom just as the Castle dominates the Magic Kingdom or the Sphere
does Epcot. But it's rendered in historically correct style and could
probably fool anyone who didn't have a Ph.D.~in Indian art history.

The next panel shows a mustachioed H. sapiens chopping down the Tree of
Life with a scimitar, and the animals fleeing every which way. The one
after that shows the misguided human getting walloped by a tidal wave,
part of a latter-day Deluge presumably brought on by his stupidity.

The final panel, then, portrays the Sapling of Life beginning to grow
back, but now Man has ditched the edged weapon and joined the other
animals in standing around to adore and praise it.

It is, in other words, a prophecy of the Bottleneck: the scenario,
commonly espoused among modern-day environmentalists, that the world
faces an upcoming period of grave ecological tribulations that will last
for a few decades or centuries and end when we find a new harmonious
modus vivendi with Nature.

Taken as a whole the frieze is a pretty brilliant piece of work.
Obviously it's not an ancient Indian ruin, and some person or people now
living deserve credit for it. But there are no signatures on the
Maharajah's game reserve at Disney World. There are no signatures on
anything, because it would ruin the whole effect to have long strings of
production credits dangling from every custom-worn brick, as they do
from Hollywood movies.

Among Hollywood writers, Disney has the reputation of being a real
wicked stepmother. It's not hard to see why. Disney is in the business
of putting out a product of seamless illusion---a magic mirror that
reflects the world back better than it really is. But a writer is
literally talking to his or her readers, not just creating an ambience
or presenting them with something to look at; and just as the
command-line interface opens a much more direct and explicit channel
from user to machine than the GUI, so it is with words, writer, and
reader.

The word, in the end, is the only system of encoding thoughts---the only
medium---that is not fungible, that refuses to dissolve in the devouring
torrent of electronic media (the richer tourists at Disney World wear
t-shirts printed with the names of famous designers, because designs
themselves can be bootlegged easily and with impunity. The only way to
make clothing that cannot be legally bootlegged is to print copyrighted
and trademarked words on it; once you have taken that step, the clothing
itself doesn't really matter, and so a t-shirt is as good as anything
else. T-shirts with expensive words on them are now the insignia of the
upper class. T-shirts with cheap words, or no words at all, are for the
commoners).

But this special quality of words and of written communication would
have the same effect on Disney's product as spray-painted graffiti on a
magic mirror. So Disney does most of its communication without resorting
to words, and for the most part, the words aren't missed. Some of
Disney's older properties, such as Peter Pan, Winnie the Pooh, and Alice
in Wonderland, came out of books. But the authors' names are rarely if
ever mentioned, and you can't buy the original books at the Disney
store. If you could, they would all seem old and queer, like very bad
knockoffs of the purer, more authentic Disney versions. Compared to more
recent productions like Beauty and the Beast and Mulan, the Disney
movies based on these books (particularly Alice in Wonderland and Peter
Pan) seem deeply bizarre, and not wholly appropriate for children. That
stands to reason, because Lewis Carroll and J.M. Barrie were very
strange men, and such is the nature of the written word that their
personal strangeness shines straight through all the layers of
Disneyfication like x-rays through a wall. Probably for this very
reason, Disney seems to have stopped buying books altogether, and now
finds its themes and characters in folk tales, which have the lapidary,
time-worn quality of the ancient bricks in the Maharajah's ruins.

If I can risk a broad generalization, most of the people who go to
Disney World have zero interest in absorbing new ideas from books. Which
sounds snide, but listen: they have no qualms about being presented with
ideas in other forms. Disney World is stuffed with environmental
messages now, and the guides at Animal Kingdom can talk your ear off
about biology.

If you followed those tourists home, you might find art, but it would be
the sort of unsigned folk art that's for sale in Disney World's African-
and Asian-themed stores. In general they only seem comfortable with
media that have been ratified by great age, massive popular acceptance,
or both.

In this world, artists are like the anonymous, illiterate stone carvers
who built the great cathedrals of Europe and then faded away into
unmarked graves in the churchyard. The cathedral as a whole is awesome
and stirring in spite, and possibly because, of the fact that we have no
idea who built it. When we walk through it we are communing not with
individual stone carvers but with an entire culture.

Disney World works the same way. If you are an intellectual type, a
reader or writer of books, the nicest thing you can say about this is
that the execution is superb. But it's easy to find the whole
environment a little creepy, because something is missing: the
translation of all its content into clear explicit written words, the
attribution of the ideas to specific people. You can't argue with it. It
seems as if a hell of a lot might be being glossed over, as if Disney
World might be putting one over on us, and possibly getting away with
all kinds of buried assumptions and muddled thinking.

But this is precisely the same as what is lost in the transition from
the command-line interface to the GUI.

Disney and Apple/Microsoft are in the same business: short-circuiting
laborious, explicit verbal communication with expensively designed
interfaces. Disney is a sort of user interface unto itself---and more
than just graphical. Let's call it a Sensorial Interface. It can be
applied to anything in the world, real or imagined, albeit at staggering
expense.

Why are we rejecting explicit word-based interfaces, and embracing
graphical or sensorial ones---a trend that accounts for the success of
both Microsoft and Disney?

Part of it is simply that the world is very complicated now---much more
complicated than the hunter-gatherer world that our brains evolved to
cope with---and we simply can't handle all of the details. We have to
delegate. We have no choice but to trust some nameless artist at Disney
or programmer at Apple or Microsoft to make a few choices for us, close
off some options, and give us a conveniently packaged executive summary.

But more importantly, it comes out of the fact that, during this
century, intellectualism failed, and everyone knows it. In places like
Russia and Germany, the common people agreed to loosen their grip on
traditional folkways, mores, and religion, and let the intellectuals run
with the ball, and they screwed everything up and turned the century
into an abbatoir. Those wordy intellectuals used to be merely tedious;
now they seem kind of dangerous as well.

We Americans are the only ones who didn't get creamed at some point
during all of this. We are free and prosperous because we have inherited
political and values systems fabricated by a particular set of
eighteenth-century intellectuals who happened to get it right. But we
have lost touch with those intellectuals, and with anything like
intellectualism, even to the point of not reading books any more, though
we are literate. We seem much more comfortable with propagating those
values to future generations nonverbally, through a process of being
steeped in media. Apparently this actually works to some degree, for
police in many lands are now complaining that local arrestees are
insisting on having their Miranda rights read to them, just like perps
in American TV cop shows. When it's explained to them that they are in a
different country, where those rights do not exist, they become
outraged. Starsky and Hutch reruns, dubbed into diverse languages, may
turn out, in the long run, to be a greater force for human rights than
the Declaration of Independence.

A huge, rich, nuclear-tipped culture that propagates its core values
through media steepage seems like a bad idea. There is an obvious risk
of running astray here. Words are the only immutable medium we have,
which is why they are the vehicle of choice for extremely important
concepts like the Ten Commandments, the Koran, and the Bill of Rights.
Unless the messages conveyed by our media are somehow pegged to a fixed,
written set of precepts, they can wander all over the place and possibly
dump loads of crap into people's minds.

Orlando used to have a military installation called McCoy Air Force
Base, with long runways from which B--52s could take off and reach Cuba,
or just about anywhere else, with loads of nukes. But now McCoy has been
scrapped and repurposed. It has been absorbed into Orlando's civilian
airport. The long runways are being used to land 747-loads of tourists
from Brazil, Italy, Russia and Japan, so that they can come to Disney
World and steep in our media for a while.

To traditional cultures, especially word-based ones such as Islam, this
is infinitely more threatening than the B--52s ever were. It is obvious,
to everyone outside of the United States, that our arch-buzzwords,
multiculturalism and diversity, are false fronts that are being used (in
many cases unwittingly) to conceal a global trend to eradicate cultural
differences. The basic tenet of multiculturalism (or ``honoring
diversity'' or whatever you want to call it) is that people need to stop
judging each other-to stop asserting (and, eventually, to stop
believing) that this is right and that is wrong, this true and that
false, one thing ugly and another thing beautiful, that God exists and
has this or that set of qualities.

The lesson most people are taking home from the Twentieth Century is
that, in order for a large number of different cultures to coexist
peacefully on the globe (or even in a neighborhood) it is necessary for
people to suspend judgment in this way. Hence (I would argue) our
suspicion of, and hostility towards, all authority figures in modern
culture. As David Foster Wallace has explained in his essay ``E Unibus
Pluram,'' this is the fundamental message of television; it is the
message that people take home, anyway, after they have steeped in our
media long enough. It's not expressed in these highfalutin terms, of
course. It comes through as the presumption that all authority
figures---teachers, generals, cops, ministers, politicians---are
hypocritical buffoons, and that hip jaded coolness is the only way to
be.

The problem is that once you have done away with the ability to make
judgments as to right and wrong, true and false, etc., there's no real
culture left. All that remains is clog dancing and macrame. The ability
to make judgments, to believe things, is the entire it point of having a
culture. I think this is why guys with machine guns sometimes pop up in
places like Luxor, and begin pumping bullets into Westerners. They
perfectly understand the lesson of McCoy Air Force Base. When their sons
come home wearing Chicago Bulls caps with the bills turned sideways, the
dads go out of their minds.

The global anti-culture that has been conveyed into every cranny of the
world by television is a culture unto itself, and by the standards of
great and ancient cultures like Islam and France, it seems grossly
inferior, at least at first. The only good thing you can say about it is
that it makes world wars and Holocausts less likely---and that is
actually a pretty good thing!

The only real problem is that anyone who has no culture, other than this
global monoculture, is completely screwed. Anyone who grows up watching
TV, never sees any religion or philosophy, is raised in an atmosphere of
moral relativism, learns about civics from watching bimbo eruptions on
network TV news, and attends a university where postmodernists vie to
outdo each other in demolishing traditional notions of truth and
quality, is going to come out into the world as one pretty feckless
human being. And---again---perhaps the goal of all this is to make us
feckless so we won't nuke each other.

On the other hand, if you are raised within some specific culture, you
end up with a basic set of tools that you can use to think about and
understand the world. You might use those tools to reject the culture
you were raised in, but at least you've got some tools.

In this country, the people who run things---who populate major law
firms and corporate boards---understand all of this at some level. They
pay lip service to multiculturalism and diversity and
non-judgmentalness, but they don't raise their own children that way. I
have highly educated, technically sophisticated friends who have moved
to small towns in Iowa to live and raise their children, and there are
Hasidic Jewish enclaves in New York where large numbers of kids are
being brought up according to traditional beliefs. Any suburban
community might be thought of as a place where people who hold certain
(mostly implicit) beliefs go to live among others who think the same
way.

And not only do these people feel some responsibility to their own
children, but to the country as a whole. Some of the upper class are
vile and cynical, of course, but many spend at least part of their time
fretting about what direction the country is going in, and what
responsibilities they have. And so issues that are important to
book-reading intellectuals, such as global environmental collapse,
eventually percolate through the porous buffer of mass culture and show
up as ancient Hindu ruins in Orlando.

You may be asking: what the hell does all this have to do with operating
systems? As I've explained, there is no way to explain the domination of
the OS market by Apple/Microsoft without looking to cultural
explanations, and so I can't get anywhere, in this essay, without first
letting you know where I'm coming from vis-a-vis contemporary culture.

Contemporary culture is a two-tiered system, like the Morlocks and the
Eloi in H.G. Wells's The Time Machine, except that it's been turned
upside down. In The Time Machine the Eloi were an effete upper class,
supported by lots of subterranean Morlocks who kept the technological
wheels turning. But in our world it's the other way round. The Morlocks
are in the minority, and they are running the show, because they
understand how everything works. The much more numerous Eloi learn
everything they know from being steeped from birth in electronic media
directed and controlled by book-reading Morlocks. So many ignorant
people could be dangerous if they got pointed in the wrong direction,
and so we've evolved a popular culture that is (a) almost unbelievably
infectious and (b) neuters every person who gets infected by it, by
rendering them unwilling to make judgments and incapable of taking
stands.

Morlocks, who have the energy and intelligence to comprehend details, go
out and master complex subjects and produce Disney-like Sensorial
Interfaces so that Eloi can get the gist without having to strain their
minds or endure boredom. Those Morlocks will go to India and tediously
explore a hundred ruins, then come home and built sanitary bug-free
versions: highlight films, as it were. This costs a lot, because
Morlocks insist on good coffee and first-class airline tickets, but
that's no problem because Eloi like to be dazzled and will gladly pay
for it all.

Now I realize that most of this probably sounds snide and bitter to the
point of absurdity: your basic snotty intellectual throwing a tantrum
about those unlettered philistines. As if I were a self-styled Moses,
coming down from the mountain all alone, carrying the stone tablets
bearing the Ten Commandments carved in immutable stone---the original
command-line interface---and blowing his stack at the weak,
unenlightened Hebrews worshipping images. Not only that, but it sounds
like I'm pumping some sort of conspiracy theory.

But that is not where I'm going with this. The situation I describe,
here, could be bad, but doesn't have to be bad and isn't necessarily bad
now:

It simply is the case that we are way too busy, nowadays, to comprehend
everything in detail. And it's better to comprehend it dimly, through an
interface, than not at all. Better for ten million Eloi to go on the
Kilimanjaro Safari at Disney World than for a thousand cardiovascular
surgeons and mutual fund managers to go on ``real'' ones in Kenya. The
boundary between these two classes is more porous than I've made it
sound. I'm always running into regular dudes---construction workers,
auto mechanics, taxi drivers, galoots in general---who were largely
aliterate until something made it necessary for them to become readers
and start actually thinking about things. Perhaps they had to come to
grips with alcoholism, perhaps they got sent to jail, or came down with
a disease, or suffered a crisis in religious faith, or simply got bored.
Such people can get up to speed on particular subjects quite rapidly.
Sometimes their lack of a broad education makes them over-apt to go off
on intellectual wild goose chases, but, hey, at least a wild goose chase
gives you some exercise. The spectre of a polity controlled by the fads
and whims of voters who actually believe that there are significant
differences between Bud Lite and Miller Lite, and who think that
professional wrestling is for real, is naturally alarming to people who
don't. But then countries controlled via the command-line interface, as
it were, by double-domed intellectuals, be they religious or secular,
are generally miserable places to live. Sophisticated people deride
Disneyesque entertainments as pat and saccharine, but, hey, if the
result of that is to instill basically warm and sympathetic reflexes, at
a preverbal level, into hundreds of millions of unlettered
media-steepers, then how bad can it be? We killed a lobster in our
kitchen last night and my daughter cried for an hour. The Japanese, who
used to be just about the fiercest people on earth, have become
infatuated with cuddly adorable cartoon characters. My own family---the
people I know best---is divided about evenly between people who will
probably read this essay and people who almost certainly won't, and I
can't say for sure that one group is necessarily warmer, happier, or
better-adjusted than the other.

\section{MORLOCKS AND ELOI AT THE KEYBOARD}

Back in the days of the command-line interface, users were all Morlocks
who had to convert their thoughts into alphanumeric symbols and type
them in, a grindingly tedious process that stripped away all ambiguity,
laid bare all hidden assumptions, and cruelly punished laziness and
imprecision. Then the interface-makers went to work on their GUIs, and
introduced a new semiotic layer between people and machines. People who
use such systems have abdicated the responsibility, and surrendered the
power, of sending bits directly to the chip that's doing the arithmetic,
and handed that responsibility and power over to the OS. This is
tempting because giving clear instructions, to anyone or anything, is
difficult. We cannot do it without thinking, and depending on the
complexity of the situation, we may have to think hard about abstract
things, and consider any number of ramifications, in order to do a good
job of it. For most of us, this is hard work. We want things to be
easier. How badly we want it can be measured by the size of Bill Gates's
fortune.

The OS has (therefore) become a sort of intellectual labor-saving device
that tries to translate humans' vaguely expressed intentions into bits.
In effect we are asking our computers to shoulder responsibilities that
have always been considered the province of human beings---we want them
to understand our desires, to anticipate our needs, to foresee
consequences, to make connections, to handle routine chores without
being asked, to remind us of what we ought to be reminded of while
filtering out noise.

At the upper (which is to say, closer to the user) levels, this is done
through a set of conventions---menus, buttons, and so on. These work in
the sense that analogies work: they help Eloi understand abstract or
unfamiliar concepts by likening them to something known. But the loftier
word ``metaphor'' is used.

The overarching concept of the MacOS was the ``desktop metaphor'' and it
subsumed any number of lesser (and frequently conflicting, or at least
mixed) metaphors. Under a GUI, a file (frequently called ``document'')
is metaphrased as a window on the screen (which is called a
``desktop''). The window is almost always too small to contain the
document and so you ``move around,'' or, more pretentiously,
``navigate'' in the document by ``clicking and dragging'' the ``thumb''
on the ``scroll bar.'' When you ``type'' (using a keyboard) or ``draw''
(using a ``mouse'') into the ``window'' or use pull-down ``menus'' and
``dialog boxes'' to manipulate its contents, the results of your labors
get stored (at least in theory) in a ``file,'' and later you can pull
the same information back up into another ``window.'' When you don't
want it anymore, you ``drag'' it into the ``trash.''

There is massively promiscuous metaphor-mixing going on here, and I
could deconstruct it 'til the cows come home, but I won't. Consider only
one word: ``document.'' When we document something in the real world, we
make fixed, permanent, immutable records of it. But computer documents
are volatile, ephemeral constellations of data. Sometimes (as when
you've just opened or saved them) the document as portrayed in the
window is identical to what is stored, under the same name, in a file on
the disk, but other times (as when you have made changes without saving
them) it is completely different. In any case, every time you hit
``Save'' you annihilate the previous version of the ``document'' and
replace it with whatever happens to be in the window at the moment. So
even the word ``save'' is being used in a sense that is grotesquely
misleading---``destroy one version, save another'' would be more
accurate.

Anyone who uses a word processor for very long inevitably has the
experience of putting hours of work into a long document and then losing
it because the computer crashes or the power goes out. Until the moment
that it disappears from the screen, the document seems every bit as
solid and real as if it had been typed out in ink on paper. But in the
next moment, without warning, it is completely and irretrievably gone,
as if it had never existed. The user is left with a feeling of
disorientation (to say nothing of annoyance) stemming from a kind of
metaphor shear---you realize that you've been living and thinking inside
of a metaphor that is essentially bogus.

So GUIs use metaphors to make computing easier, but they are bad
metaphors. Learning to use them is essentially a word game, a process of
learning new definitions of words like ``window'' and ``document'' and
``save'' that are different from, and in many cases almost diametrically
opposed to, the old. Somewhat improbably, this has worked very well, at
least from a commercial standpoint, which is to say that Apple/Microsoft
have made a lot of money off of it. All of the other modern operating
systems have learned that in order to be accepted by users they must
conceal their underlying gutwork beneath the same sort of spackle. This
has some advantages: if you know how to use one GUI operating system,
you can probably work out how to use any other in a few minutes.
Everything works a little differently, like European plumbing---but with
some fiddling around, you can type a memo or surf the web.

Most people who shop for OSes (if they bother to shop at all) are
comparing not the underlying functions but the superficial look and
feel. The average buyer of an OS is not really paying for, and is not
especially interested in, the low-level code that allocates memory or
writes bytes onto the disk. What we're really buying is a system of
metaphors. And---much more important---what we're buying into is the
underlying assumption that metaphors are a good way to deal with the
world.

Recently a lot of new hardware has become available that gives computers
numerous interesting ways of affecting the real world: making paper spew
out of printers, causing words to appear on screens thousands of miles
away, shooting beams of radiation through cancer patients, creating
realistic moving pictures of the Titanic. Windows is now used as an OS
for cash registers and bank tellers' terminals. My satellite TV system
uses a sort of GUI to change channels and show program guides. Modern
cellular telephones have a crude GUI built into a tiny LCD screen. Even
Legos now have a GUI: you can buy a Lego set called Mindstorms that
enables you to build little Lego robots and program them through a GUI
on your computer.

So we are now asking the GUI to do a lot more than serve as a glorified
typewriter. Now we want to become a generalized tool for dealing with
reality. This has become a bonanza for companies that make a living out
of bringing new technology to the mass market.

Obviously you cannot sell a complicated technological system to people
without some sort of interface that enables them to use it. The internal
combustion engine was a technological marvel in its day, but useless as
a consumer good until a clutch, transmission, steering wheel and
throttle were connected to it. That odd collection of gizmos, which
survives to this day in every car on the road, made up what we would
today call a user interface. But if cars had been invented after
Macintoshes, carmakers would not have bothered to gin up all of these
arcane devices. We would have a computer screen instead of a dashboard,
and a mouse (or at best a joystick) instead of a steering wheel, and
we'd shift gears by pulling down a menu:

\begin{lstlisting}
PARK --- REVERSE --- NEUTRAL ---- 3 2 1 --- Help...
\end{lstlisting}
A few lines of computer code can thus be made to substitute for any
imaginable mechanical interface. The problem is that in many cases the
substitute is a poor one. Driving a car through a GUI would be a
miserable experience. Even if the GUI were perfectly bug-free, it would
be incredibly dangerous, because menus and buttons simply can't be as
responsive as direct mechanical controls. My friend's dad, the gentleman
who was restoring the MGB, never would have bothered with it if it had
been equipped with a GUI. It wouldn't have been any fun.

The steering wheel and gearshift lever were invented during an era when
the most complicated technology in most homes was a butter churn. Those
early carmakers were simply lucky, in that they could dream up whatever
interface was best suited to the task of driving an automobile, and
people would learn it. Likewise with the dial telephone and the AM
radio. By the time of the Second World War, most people knew several
interfaces: they could not only churn butter but also drive a car, dial
a telephone, turn on a radio, summon flame from a cigarette lighter, and
change a light bulb.

But now every little thing---wristwatches, VCRs, stoves---is jammed with
features, and every feature is useless without an interface. If you are
like me, and like most other consumers, you have never used ninety
percent of the available features on your microwave oven, VCR, or
cellphone. You don't even know that these features exist. The small
benefit they might bring you is outweighed by the sheer hassle of having
to learn about them. This has got to be a big problem for makers of
consumer goods, because they can't compete without offering features.

It's no longer acceptable for engineers to invent a wholly novel user
interface for every new product, as they did in the case of the
automobile, partly because it's too expensive and partly because
ordinary people can only learn so much. If the VCR had been invented a
hundred years ago, it would have come with a thumbwheel to adjust the
tracking and a gearshift to change between forward and reverse and a big
cast-iron handle to load or to eject the cassettes. It would have had a
big analog clock on the front of it, and you would have set the time by
moving the hands around on the dial. But because the VCR was invented
when it was---during a sort of awkward transitional period between the
era of mechanical interfaces and GUIs---it just had a bunch of
pushbuttons on the front, and in order to set the time you had to push
the buttons in just the right way. This must have seemed reasonable
enough to the engineers responsible for it, but to many users it was
simply impossible. Thus the famous blinking 12:00 that appears on so
many VCRs. Computer people call this ``the blinking twelve problem''.
When they talk about it, though, they usually aren't talking about VCRs.

Modern VCRs usually have some kind of on-screen programming, which means
that you can set the time and control other features through a sort of
primitive GUI. GUIs have virtual pushbuttons too, of course, but they
also have other types of virtual controls, like radio buttons,
checkboxes, text entry boxes, dials, and scrollbars. Interfaces made out
of these components seem to be a lot easier, for many people, than
pushing those little buttons on the front of the machine, and so the
blinking 12:00 itself is slowly disappearing from America's living
rooms. The blinking twelve problem has moved on to plague other
technologies.

So the GUI has gone beyond being an interface to personal computers, and
become a sort of meta-interface that is pressed into service for every
new piece of consumer technology. It is rarely an ideal fit, but having
an ideal, or even a good interface is no longer the priority; the
important thing now is having some kind of interface that customers will
actually use, so that manufacturers can claim, with a straight face,
that they are offering new features.

We want GUIs largely because they are convenient and because they are
easy--- or at least the GUI makes it seem that way Of course, nothing is
really easy and simple, and putting a nice interface on top of it does
not change that fact. A car controlled through a GUI would be easier to
drive than one controlled through pedals and steering wheel, but it
would be incredibly dangerous.

By using GUIs all the time we have insensibly bought into a premise that
few people would have accepted if it were presented to them bluntly:
namely, that hard things can be made easy, and complicated things
simple, by putting the right interface on them. In order to understand
how bizarre this is, imagine that book reviews were written according to
the same values system that we apply to user interfaces: ``The writing
in this book is marvelously simple-minded and glib; the author glosses
over complicated subjects and employs facile generalizations in almost
every sentence. Readers rarely have to think, and are spared all of the
difficulty and tedium typically involved in reading old-fashioned
books.'' As long as we stick to simple operations like setting the
clocks on our VCRs, this is not so bad. But as we try to do more
ambitious things with our technologies, we inevitably run into the
problem of:

\section{METAPHOR SHEAR}

I began using Microsoft Word as soon as the first version was released
around 1985. After some initial hassles I found it to be a better tool
than MacWrite, which was its only competition at the time. I wrote a lot
of stuff in early versions of Word, storing it all on floppies, and
transferred the contents of all my floppies to my first hard drive,
which I acquired around 1987. As new versions of Word came out I
faithfully upgraded, reasoning that as a writer it made sense for me to
spend a certain amount of money on tools.

Sometime in the mid--1980's I attempted to open one of my old,
circa--1985 Word documents using the version of Word then current: 6.0
It didn't work. Word 6.0 did not recognize a document created by an
earlier version of itself. By opening it as a text file, I was able to
recover the sequences of letters that made up the text of the document.
My words were still there. But the formatting had been run through a log
chipper---the words I'd written were interrupted by spates of empty
rectangular boxes and gibberish.

Now, in the context of a business (the chief market for Word) this sort
of thing is only an annoyance---one of the routine hassles that go along
with using computers. It's easy to buy little file converter programs
that will take care of this problem. But if you are a writer whose
career is words, whose professional identity is a corpus of written
documents, this kind of thing is extremely disquieting. There are very
few fixed assumptions in my line of work, but one of them is that once
you have written a word, it is written, and cannot be unwritten. The ink
stains the paper, the chisel cuts the stone, the stylus marks the clay,
and something has irrevocably happened (my brother-in-law is a
theologian who reads 3250-year-old cuneiform tablets---he can recognize
the handwriting of particular scribes, and identify them by name). But
word-processing software---particularly the sort that employs special,
complex file formats---has the eldritch power to unwrite things. A small
change in file formats, or a few twiddled bits, and months' or years'
literary output can cease to exist.

Now this was technically a fault in the application (Word 6.0 for the
Macintosh) not the operating system (MacOS 7 point something) and so the
initial target of my annoyance was the people who were responsible for
Word. But. On the other hand, I could have chosen the ``save as text''
option in Word and saved all of my documents as simple telegrams, and
this problem would not have arisen. Instead I had allowed myself to be
seduced by all of those flashy formatting options that hadn't even
existed until GUIs had come along to make them practicable. I had gotten
into the habit of using them to make my documents look pretty (perhaps
prettier than they deserved to look; all of the old documents on those
floppies turned out to be more or less crap). Now I was paying the price
for that self-indulgence. Technology had moved on and found ways to make
my documents look even prettier, and the consequence of it was that all
old ugly documents had ceased to exist.

It was---if you'll pardon me for a moment's strange little fantasy---as
if I'd gone to stay at some resort, some exquisitely designed and
art-directed hotel, placing myself in the hands of past masters of the
Sensorial Interface, and had sat down in my room and written a story in
ballpoint pen on a yellow legal pad, and when I returned from dinner,
discovered that the maid had taken my work away and left behind in its
place a quill pen and a stack of fine parchment---explaining that the
room looked ever so much finer this way, and it was all part of a
routine upgrade. But written on these sheets of paper, in flawless
penmanship, were long sequences of words chosen at random from the
dictionary. Appalling, sure, but I couldn't really lodge a complaint
with the management, because by staying at this resort I had given my
consent to it. I had surrendered my Morlock credentials and become an
Eloi.

\section{LINUX}

During the late 1980's and early 1990's I spent a lot of time
programming Macintoshes, and eventually decided for fork over several
hundred dollars for an Apple product called the Macintosh Programmer's
Workshop, or MPW. MPW had competitors, but it was unquestionably the
premier software development system for the Mac. It was what Apple's own
engineers used to write Macintosh code. Given that MacOS was far more
technologically advanced, at the time, than its competition, and that
Linux did not even exist yet, and given that this was the actual program
used by Apple's world-class team of creative engineers, I had high
expectations. It arrived on a stack of floppy disks about a foot high,
and so there was plenty of time for my excitement to build during the
endless installation process. The first time I launched MPW, I was
probably expecting some kind of touch-feely multimedia showcase. Instead
it was austere, almost to the point of being intimidating. It was a
scrolling window into which you could type simple, unformatted text. The
system would then interpret these lines of text as commands, and try to
execute them.

It was, in other words, a glass teletype running a command line
interface. It came with all sorts of cryptic but powerful commands,
which could be invoked by typing their names, and which I learned to use
only gradually. It was not until a few years later, when I began messing
around with Unix, that I understood that the command line interface
embodied in MPW was a re-creation of Unix.

In other words, the first thing that Apple's hackers had done when
they'd got the MacOS up and running---probably even before they'd gotten
it up and running---was to re-create the Unix interface, so that they
would be able to get some useful work done. At the time, I simply
couldn't get my mind around this, but: as far as Apple's hackers were
concerned, the Mac's vaunted Graphical User Interface was an impediment,
something to be circumvented before the little toaster even came out
onto the market.

Even before my Powerbook crashed and obliterated my big file in July
1995, there had been danger signs. An old college buddy of mine, who
starts and runs high-tech companies in Boston, had developed a
commercial product using Macintoshes as the front end. Basically the
Macs were high-performance graphics terminals, chosen for their sweet
user interface, giving users access to a large database of graphical
information stored on a network of much more powerful, but less
user-friendly, computers. This fellow was the second person who turned
me on to Macintoshes, by the way, and through the mid--1980's we had
shared the thrill of being high-tech cognoscenti, using superior Apple
technology in a world of DOS-using knuckleheads. Early versions of my
friend's system had worked well, he told me, but when several machines
joined the network, mysterious crashes began to occur; sometimes the
whole network would just freeze. It was one of those bugs that could not
be reproduced easily. Finally they figured out that these network
crashes were triggered whenever a user, scanning the menus for a
particular item, held down the mouse button for more than a couple of
seconds.

Fundamentally, the MacOS could only do one thing at a time. Drawing a
menu on the screen is one thing. So when a menu was pulled down, the
Macintosh was not capable of doing anything else until that indecisive
user released the button.

This is not such a bad thing in a single-user, single-process machine
(although it's a fairly bad thing), but it's no good in a machine that
is on a network, because being on a network implies some kind of
continual low-level interaction with other machines. By failing to
respond to the network, the Mac caused a network-wide crash.

In order to work with other computers, and with networks, and with
various different types of hardware, an OS must be incomparably more
complicated and powerful than either MS-DOS or the original MacOS. The
only way of connecting to the Internet that's worth taking seriously is
PPP, the Point-to-Point Protocol, which (never mind the details) makes
your computer---temporarily---a full-fledged member of the Global
Internet, with its own unique address, and various privileges, powers,
and responsibilities appertaining thereunto. Technically it means your
machine is running the TCP/IP protocol, which, to make a long story
short, revolves around sending packets of data back and forth, in no
particular order, and at unpredictable times, according to a clever and
elegant set of rules. But sending a packet of data is one thing, and so
an OS that can only do one thing at a time cannot simultaneously be part
of the Internet and do anything else. When TCP/IP was invented, running
it was an honor reserved for Serious Computers---mainframes and
high-powered minicomputers used in technical and commercial
settings---and so the protocol is engineered around the assumption that
every computer using it is a serious machine, capable of doing many
things at once. Not to put too fine a point on it, a Unix machine.
Neither MacOS nor MS-DOS was originally built with that in mind, and so
when the Internet got hot, radical changes had to be made.

When my Powerbook broke my heart, and when Word stopped recognizing my
old files, I jumped to Unix. The obvious alternative to MacOS would have
been Windows. I didn't really have anything against Microsoft, or
Windows. But it was pretty obvious, now, that old PC operating systems
were overreaching, and showing the strain, and, perhaps, were best
avoided until they had learned to walk and chew gum at the same time.

The changeover took place on a particular day in the summer of 1995. I
had been San Francisco for a couple of weeks, using my PowerBook to work
on a document. The document was too big to fit onto a single floppy, and
so I hadn't made a backup since leaving home. The PowerBook crashed and
wiped out the entire file.

It happened just as I was on my way out the door to visit a company
called Electric Communities, which in those days was in Los Altos. I
took my PowerBook with me. My friends at Electric Communities were Mac
users who had all sorts of utility software for unerasing files and
recovering from disk crashes, and I was certain I could get most of the
file back.

As it turned out, two different Mac crash recovery utilities were unable
to find any trace that my file had ever existed. It was completely and
systematically wiped out. We went through that hard disk block by block
and found disjointed fragments of countless old, discarded, forgotten
files, but none of what I wanted. The metaphor shear was especially
brutal that day. It was sort of like watching the girl you've been in
love with for ten years get killed in a car wreck, and then attending
her autopsy, and learning that underneath the clothes and makeup she was
just flesh and blood.

I must have been reeling around the offices of Electric Communities in
some kind of primal Jungian fugue, because at this moment three weirdly
synchronistic things happened.

\begin{enumerate}[1.]
\item
  Randy Farmer, a co-founder of the company, came in for a quick visit
  along with his family---he was recovering from back surgery at the
  time. He had some hot gossip: ``Windows 95 mastered today.'' What this
  meant was that Microsoft's new operating system had, on this day, been
  placed on a special compact disk known as a golden master, which would
  be used to stamp out a jintillion copies in preparation for its
  thunderous release a few weeks later. This news was received peevishly
  by the staff of Electric Communities, including one whose office door
  was plastered with the usual assortment of cartoons and novelties,
  e.g.
\item
  a copy of a Dilbert cartoon in which Dilbert, the long-suffering
  corporate software engineer, encounters a portly, bearded, hairy man
  of a certain age---a bit like Santa Claus, but darker, with a certain
  edge about him. Dilbert recognizes this man, based upon his appearance
  and affect, as a Unix hacker, and reacts with a certain mixture of
  nervousness, awe, and hostility. Dilbert jabs weakly at the disturbing
  interloper for a couple of frames; the Unix hacker listens with a kind
  of infuriating, beatific calm, then, in the last frame, reaches into
  his pocket. ``Here's a nickel, kid,'' he says, ``go buy yourself a
  real computer.''
\item
  the owner of the door, and the cartoon, was one Doug Barnes. Barnes
  was known to harbor certain heretical opinions on the subject of
  operating systems. Unlike most Bay Area techies who revered the
  Macintosh, considering it to be a true hacker's machine, Barnes was
  fond of pointing out that the Mac, with its hermetically sealed
  architecture, was actually hostile to hackers, who are prone to
  tinkering and dogmatic about openness. By contrast, the IBM-compatible
  line of machines, which can easily be taken apart and plugged back
  together, was much more hackable.
\end{enumerate}
So when I got home I began messing around with Linux, which is one of
many, many different concrete implementations of the abstract, Platonic
ideal called Unix. I was not looking forward to changing over to a new
OS, because my credit cards were still smoking from all the money I'd
spent on Mac hardware over the years. But Linux's great virtue was, and
is, that it would run on exactly the same sort of hardware as the
Microsoft OSes---which is to say, the cheapest hardware in existence. As
if to demonstrate why this was a great idea, I was, within a week or two
of returning home, able to get my hand on a then-decent computer (a
33-MHz 486 box) for free, because I knew a guy who worked in an office
where they were simply being thrown away. Once I got it home, I yanked
the hood off, stuck my hands in, and began switching cards around. If
something didn't work, I went to a used-computer outlet and pawed
through a bin full of components and bought a new card for a few bucks.

The availability of all this cheap but effective hardware was an
unintended consequence of decisions that had been made more than a
decade earlier by IBM and Microsoft. When Windows came out, and brought
the GUI to a much larger market, the hardware regime changed: the cost
of color video cards and high-resolution monitors began to drop, and is
dropping still. This free-for-all approach to hardware meant that
Windows was unavoidably clunky compared to MacOS. But the GUI brought
computing to such a vast audience that volume went way up and prices
collapsed. Meanwhile Apple, which so badly wanted a clean, integrated OS
with video neatly integrated into processing hardware, had fallen far
behind in market share, at least partly because their beautiful hardware
cost so much.

But the price that we Mac owners had to pay for superior aesthetics and
engineering was not merely a financial one. There was a cultural price
too, stemming from the fact that we couldn't open up the hood and mess
around with it. Doug Barnes was right. Apple, in spite of its reputation
as the machine of choice of scruffy, creative hacker types, had actually
created a machine that discouraged hacking, while Microsoft, viewed as a
technological laggard and copycat, had created a vast, disorderly parts
bazaar---a primordial soup that eventually self-assembled into Linux.

\section{THE HOLE HAWG OF OPERATING SYSTEMS}

Unix has always lurked provocatively in the background of the operating
system wars, like the Russian Army. Most people know it only by
reputation, and its reputation, as the Dilbert cartoon suggests, is
mixed. But everyone seems to agree that if it could only get its act
together and stop surrendering vast tracts of rich agricultural land and
hundreds of thousands of prisoners of war to the onrushing invaders, it
could stomp them (and all other opposition) flat.

It is difficult to explain how Unix has earned this respect without
going into mind-smashing technical detail. Perhaps the gist of it can be
explained by telling a story about drills.

The Hole Hawg is a drill made by the Milwaukee Tool Company. If you look
in a typical hardware store you may find smaller Milwaukee drills but
not the Hole Hawg, which is too powerful and too expensive for
homeowners. The Hole Hawg does not have the pistol-like design of a
cheap homeowner's drill. It is a cube of solid metal with a handle
sticking out of one face and a chuck mounted in another. The cube
contains a disconcertingly potent electric motor. You can hold the
handle and operate the trigger with your index finger, but unless you
are exceptionally strong you cannot control the weight of the Hole Hawg
with one hand; it is a two-hander all the way. In order to fight off the
counter-torque of the Hole Hawg you use a separate handle (provided),
which you screw into one side of the iron cube or the other depending on
whether you are using your left or right hand to operate the trigger.
This handle is not a sleek, ergonomically designed item as it would be
in a homeowner's drill. It is simply a foot-long chunk of regular
galvanized pipe, threaded on one end, with a black rubber handle on the
other. If you lose it, you just go to the local plumbing supply store
and buy another chunk of pipe.

During the Eighties I did some construction work. One day, another
worker leaned a ladder against the outside of the building that we were
putting up, climbed up to the second-story level, and used the Hole Hawg
to drill a hole through the exterior wall. At some point, the drill bit
caught in the wall. The Hole Hawg, following its one and only
imperative, kept going. It spun the worker's body around like a rag
doll, causing him to knock his own ladder down. Fortunately he kept his
grip on the Hole Hawg, which remained lodged in the wall, and he simply
dangled from it and shouted for help until someone came along and
reinstated the ladder.

I myself used a Hole Hawg to drill many holes through studs, which it
did as a blender chops cabbage. I also used it to cut a few
six-inch-diameter holes through an old lath-and-plaster ceiling. I
chucked in a new hole saw, went up to the second story, reached down
between the newly installed floor joists, and began to cut through the
first-floor ceiling below. Where my homeowner's drill had labored and
whined to spin the huge bit around, and had stalled at the slightest
obstruction, the Hole Hawg rotated with the stupid consistency of a
spinning planet. When the hole saw seized up, the Hole Hawg spun itself
and me around, and crushed one of my hands between the steel pipe handle
and a joist, producing a few lacerations, each surrounded by a wide
corona of deeply bruised flesh. It also bent the hole saw itself, though
not so badly that I couldn't use it. After a few such run-ins, when I
got ready to use the Hole Hawg my heart actually began to pound with
atavistic terror.

But I never blamed the Hole Hawg; I blamed myself. The Hole Hawg is
dangerous because it does exactly what you tell it to. It is not bound
by the physical limitations that are inherent in a cheap drill, and
neither is it limited by safety interlocks that might be built into a
homeowner's product by a liability-conscious manufacturer. The danger
lies not in the machine itself but in the user's failure to envision the
full consequences of the instructions he gives to it.

A smaller tool is dangerous too, but for a completely different reason:
it tries to do what you tell it to, and fails in some way that is
unpredictable and almost always undesirable. But the Hole Hawg is like
the genie of the ancient fairy tales, who carries out his master's
instructions literally and precisely and with unlimited power, often
with disastrous, unforeseen consequences.

Pre-Hole Hawg, I used to examine the drill selection in hardware stores
with what I thought was a judicious eye, scorning the smaller low-end
models and hefting the big expensive ones appreciatively, wishing I
could afford one of them babies. Now I view them all with such contempt
that I do not even consider them to be real drills---merely scaled-up
toys designed to exploit the self-delusional tendencies of soft-handed
homeowners who want to believe that they have purchased an actual tool.
Their plastic casings, carefully designed and focus-group-tested to
convey a feeling of solidity and power, seem disgustingly flimsy and
cheap to me, and I am ashamed that I was ever bamboozled into buying
such knicknacks.

It is not hard to imagine what the world would look like to someone who
had been raised by contractors and who had never used any drill other
than a Hole Hawg. Such a person, presented with the best and most
expensive hardware-store drill, would not even recognize it as such. He
might instead misidentify it as a child's toy, or some kind of motorized
screwdriver. If a salesperson or a deluded homeowner referred to it as a
drill, he would laugh and tell them that they were mistaken---they
simply had their terminology wrong. His interlocutor would go away
irritated, and probably feeling rather defensive about his basement full
of cheap, dangerous, flashy, colorful tools.

Unix is the Hole Hawg of operating systems, and Unix hackers, like Doug
Barnes and the guy in the Dilbert cartoon and many of the other people
who populate Silicon Valley, are like contractor's sons who grew up
using only Hole Hawgs. They might use Apple/Microsoft OSes to write
letters, play video games, or balance their checkbooks, but they cannot
really bring themselves to take these operating systems seriously.

\section{THE ORAL TRADITION}

Unix is hard to learn. The process of learning it is one of multiple
small epiphanies. Typically you are just on the verge of inventing some
necessary tool or utility when you realize that someone else has already
invented it, and built it in, and this explains some odd file or
directory or command that you have noticed but never really understood
before.

For example there is a command (a small program, part of the OS) called
whoami, which enables you to ask the computer who it thinks you are. On
a Unix machine, you are always logged in under some name---possibly even
your own! What files you may work with, and what software you may use,
depends on your identity. When I started out using Linux, I was on a
non-networked machine in my basement, with only one user account, and so
when I became aware of the whoami command it struck me as ludicrous. But
once you are logged in as one person, you can temporarily switch over to
a pseudonym in order to access different files. If your machine is on
the Internet, you can log onto other computers, provided you have a user
name and a password. At that point the distant machine becomes no
different in practice from the one right in front of you. These changes
in identity and location can easily become nested inside each other,
many layers deep, even if you aren't doing anything nefarious. Once you
have forgotten who and where you are, the whoami command is
indispensible. I use it all the time.

The file systems of Unix machines all have the same general structure.
On your flimsy operating systems, you can create directories (folders)
and give them names like Frodo or My Stuff and put them pretty much
anywhere you like. But under Unix the highest level---the root---of the
filesystem is always designated with the single character ``/'' and it
always contains the same set of top-level directories:

\begin{lstlisting}
/usr /etc /var /bin /proc /boot /home /root /sbin /dev /lib /tmp
\end{lstlisting}
and each of these directories typically has its own distinct structure
of subdirectories. Note the obsessive use of abbreviations and avoidance
of capital letters; this is a system invented by people to whom
repetitive stress disorder is what black lung is to miners. Long names
get worn down to three-letter nubbins, like stones smoothed by a river.

This is not the place to try to explain why each of the above
directories exists, and what is contained in it. At first it all seems
obscure; worse, it seems deliberately obscure. When I started using
Linux I was accustomed to being able to create directories wherever I
wanted and to give them whatever names struck my fancy. Under Unix you
are free to do that, of course (you are free to do anything) but as you
gain experience with the system you come to understand that the
directories listed above were created for the best of reasons and that
your life will be much easier if you follow along (within /home, by the
way, you have pretty much unlimited freedom).

After this kind of thing has happened several hundred or thousand times,
the hacker understands why Unix is the way it is, and agrees that it
wouldn't be the same any other way. It is this sort of acculturation
that gives Unix hackers their confidence in the system, and the attitude
of calm, unshakable, annoying superiority captured in the Dilbert
cartoon. Windows 95 and MacOS are products, contrived by engineers in
the service of specific companies. Unix, by contrast, is not so much a
product as it is a painstakingly compiled oral history of the hacker
subculture. It is our Gilgamesh epic.

What made old epics like Gilgamesh so powerful and so long-lived was
that they were living bodies of narrative that many people knew by
heart, and told over and over again---making their own personal
embellishments whenever it struck their fancy. The bad embellishments
were shouted down, the good ones picked up by others, polished,
improved, and, over time, incorporated into the story. Likewise, Unix is
known, loved, and understood by so many hackers that it can be
re-created from scratch whenever someone needs it. This is very
difficult to understand for people who are accustomed to thinking of
OSes as things that absolutely have to be bought.

Many hackers have launched more or less successful re-implementations of
the Unix ideal. Each one brings in new embellishments. Some of them die
out quickly, some are merged with similar, parallel innovations created
by different hackers attacking the same problem, others still are
embraced, and adopted into the epic. Thus Unix has slowly accreted
around a simple kernel and acquired a kind of complexity and asymmetry
about it that is organic, like the roots of a tree, or the branchings of
a coronary artery. Understanding it is more like anatomy than physics.

For at least a year, prior to my adoption of Linux, I had been hearing
about it. Credible, well-informed people kept telling me that a bunch of
hackers had got together an implentation of Unix that could be
downloaded, free of charge, from the Internet. For a long time I could
not bring myself to take the notion seriously. It was like hearing
rumors that a group of model rocket enthusiasts had created a completely
functional Saturn V by exchanging blueprints on the Net and mailing
valves and flanges to each other.

But it's true. Credit for Linux generally goes to its human namesake,
one Linus Torvalds, a Finn who got the whole thing rolling in 1991 when
he used some of the GNU tools to write the beginnings of a Unix kernel
that could run on PC-compatible hardware. And indeed Torvalds deserves
all the credit he has ever gotten, and a whole lot more. But he could
not have made it happen by himself, any more than Richard Stallman could
have. To write code at all, Torvalds had to have cheap but powerful
development tools, and these he got from Stallman's GNU project.

And he had to have cheap hardware on which to write that code. Cheap
hardware is a much harder thing to arrange than cheap software; a single
person (Stallman) can write software and put it up on the Net for free,
but in order to make hardware it's necessary to have a whole industrial
infrastructure, which is not cheap by any stretch of the imagination.
Really the only way to make hardware cheap is to punch out an incredible
number of copies of it, so that the unit cost eventually drops. For
reasons already explained, Apple had no desire to see the cost of
hardware drop. The only reason Torvalds had cheap hardware was
Microsoft.

Microsoft refused to go into the hardware business, insisted on making
its software run on hardware that anyone could build, and thereby
created the market conditions that allowed hardware prices to plummet.
In trying to understand the Linux phenomenon, then, we have to look not
to a single innovator but to a sort of bizarre Trinity: Linus Torvalds,
Richard Stallman, and Bill Gates. Take away any of these three and Linux
would not exist.

\section{OS SHOCK}

Young Americans who leave their great big homogeneous country and visit
some other part of the world typically go through several stages of
culture shock: first, dumb wide-eyed astonishment. Then a tentative
engagement with the new country's manners, cuisine, public transit
systems and toilets, leading to a brief period of fatuous confidence
that they are instant experts on the new country. As the visit wears on,
homesickness begins to set in, and the traveler begins to appreciate,
for the first time, how much he or she took for granted at home. At the
same time it begins to seem obvious that many of one's own cultures and
traditions are essentially arbitrary, and could have been different;
driving on the right side of the road, for example. When the traveler
returns home and takes stock of the experience, he or she may have
learned a good deal more about America than about the country they went
to visit.

For the same reasons, Linux is worth trying. It is a strange country
indeed, but you don't have to live there; a brief sojourn suffices to
give some flavor of the place and---more importantly---to lay bare
everything that is taken for granted, and all that could have been done
differently, under Windows or MacOS.

You can't try it unless you install it. With any other OS, installing it
would be a straightforward transaction: in exchange for money, some
company would give you a CD-ROM, and you would be on your way. But a lot
is subsumed in that kind of transaction, and has to be gone through and
picked apart.

We like plain dealings and straightforward transactions in America. If
you go to Egypt and, say, take a taxi somewhere, you become a part of
the taxi driver's life; he refuses to take your money because it would
demean your friendship, he follows you around town, and weeps hot tears
when you get in some other guy's taxi. You end up meeting his kids at
some point, and have to devote all sort of ingenuity to finding some way
to compensate him without insulting his honor. It is exhausting.
Sometimes you just want a simple Manhattan-style taxi ride.

But in order to have an American-style setup, where you can just go out
and hail a taxi and be on your way, there must exist a whole hidden
apparatus of medallions, inspectors, commissions, and so forth---which
is fine as long as taxis are cheap and you can always get one. When the
system fails to work in some way, it is mysterious and infuriating and
turns otherwise reasonable people into conspiracy theorists. But when
the Egyptian system breaks down, it breaks down transparently. You can't
get a taxi, but your driver's nephew will show up, on foot, to explain
the problem and apologize.

Microsoft and Apple do things the Manhattan way, with vast complexity
hidden behind a wall of interface. Linux does things the Egypt way, with
vast complexity strewn about all over the landscape. If you've just
flown in from Manhattan, your first impulse will be to throw up your
hands and say ``For crying out loud! Will you people get a grip on
yourselves!?'' But this does not make friends in Linux-land any better
than it would in Egypt.

You can suck Linux right out of the air, as it were, by downloading the
right files and putting them in the right places, but there probably are
not more than a few hundred people in the world who could create a
functioning Linux system in that way. What you really need is a
distribution of Linux, which means a prepackaged set of files. But
distributions are a separate thing from Linux per se.

Linux per se is not a specific set of ones and zeroes, but a
self-organizing Net subculture. The end result of its collective
lucubrations is a vast body of source code, almost all written in C (the
dominant computer programming language). ``Source code'' just means a
computer program as typed in and edited by some hacker. If it's in C,
the file name will probably have .c or .cpp on the end of it, depending
on which dialect was used; if it's in some other language it will have
some other suffix. Frequently these sorts of files can be found in a
directory with the name /src which is the hacker's Hebraic abbreviation
of ``source.''

Source files are useless to your computer, and of little interest to
most users, but they are of gigantic cultural and political
significance, because Microsoft and Apple keep them secret while Linux
makes them public. They are the family jewels. They are the sort of
thing that in Hollywood thrillers is used as a McGuffin: the plutonium
bomb core, the top-secret blueprints, the suitcase of bearer bonds, the
reel of microfilm. If the source files for Windows or MacOS were made
public on the Net, then those OSes would become free, like Linux---only
not as good, because no one would be around to fix bugs and answer
questions. Linux is ``open source'' software meaning, simply, that
anyone can get copies of its source code files.

Your computer doesn't want source code any more than you do; it wants
object code. Object code files typically have the suffix .o and are
unreadable all but a few, highly strange humans, because they consist of
ones and zeroes. Accordingly, this sort of file commonly shows up in a
directory with the name /bin, for ``binary.''

Source files are simply ASCII text files. ASCII denotes a particular way
of encoding letters into bit patterns. In an ASCII file, each character
has eight bits all to itself. This creates a potential ``alphabet'' of
256 distinct characters, in that eight binary digits can form that many
unique patterns. In practice, of course, we tend to limit ourselves to
the familiar letters and digits. The bit-patterns used to represent
those letters and digits are the same ones that were physically punched
into the paper tape by my high school teletype, which in turn were the
same one used by the telegraph industry for decades previously. ASCII
text files, in other words, are telegrams, and as such they have no
typographical frills. But for the same reason they are eternal, because
the code never changes, and universal, because every text editing and
word processing software ever written knows about this code.

Therefore just about any software can be used to create, edit, and read
source code files. Object code files, then, are created from these
source files by a piece of software called a compiler, and forged into a
working application by another piece of software called a linker.

The triad of editor, compiler, and linker, taken together, form the core
of a software development system. Now, it is possible to spend a lot of
money on shrink-wrapped development systems with lovely graphical user
interfaces and various ergonomic enhancements. In some cases it might
even be a good and reasonable way to spend money. But on this side of
the road, as it were, the very best software is usually the free stuff.
Editor, compiler and linker are to hackers what ponies, stirrups, and
archery sets were to the Mongols. Hackers live in the saddle, and hack
on their own tools even while they are using them to create new
applications. It is quite inconceivable that superior hacking tools
could have been created from a blank sheet of paper by product
engineers. Even if they are the brightest engineers in the world they
are simply outnumbered.

In the GNU/Linux world there are two major text editing programs: the
minimalist vi (known in some implementations as elvis) and the
maximalist emacs. I use emacs, which might be thought of as a
thermonuclear word processor. It was created by Richard Stallman; enough
said. It is written in Lisp, which is the only computer language that is
beautiful. It is colossal, and yet it only edits straight ASCII text
files, which is to say, no fonts, no boldface, no underlining. In other
words, the engineer-hours that, in the case of Microsoft Word, were
devoted to features like mail merge, and the ability to embed
feature-length motion pictures in corporate memoranda, were, in the case
of emacs, focused with maniacal intensity on the deceptively
simple-seeming problem of editing text. If you are a professional
writer---i.e., if someone else is getting paid to worry about how your
words are formatted and printed---emacs outshines all other editing
software in approximately the same way that the noonday sun does the
stars. It is not just bigger and brighter; it simply makes everything
else vanish. For page layout and printing you can use TeX: a vast corpus
of typesetting lore written in C and also available on the Net for free.

I could say a lot about emacs and TeX, but right now I am trying to tell
a story about how to actually install Linux on your machine. The
hard-core survivalist approach would be to download an editor like
emacs, and the GNU Tools---the compiler and linker---which are polished
and excellent to the same degree as emacs. Equipped with these, one
would be able to start downloading ASCII source code files (/src) and
compiling them into binary object code files (/bin) that would run on
the machine. But in order to even arrive at this point---to get emacs
running, for example---you have to have Linux actually up and running on
your machine. And even a minimal Linux operating system requires
thousands of binary files all acting in concert, and arranged and linked
together just so.

Several entities have therefore taken it upon themselves to create
``distributions'' of Linux. If I may extend the Egypt analogy slightly,
these entities are a bit like tour guides who meet you at the airport,
who speak your language, and who help guide you through the initial
culture shock. If you are an Egyptian, of course, you see it the other
way; tour guides exist to keep brutish outlanders from traipsing through
your mosques and asking you the same questions over and over and over
again.

Some of these tour guides are commercial organizations, such as Red Hat
Software, which makes a Linux distribution called Red Hat that has a
relatively commercial sheen to it. In most cases you put a Red Hat
CD-ROM into your PC and reboot and it handles the rest. Just as a tour
guide in Egypt will expect some sort of compensation for his services,
commercial distributions need to be paid for. In most cases they cost
almost nothing and are well worth it.

I use a distribution called Debian (the word is a contraction of
``Deborah'' and ``Ian'') which is non-commercial. It is organized (or
perhaps I should say ``it has organized itself'') along the same lines
as Linux in general, which is to say that it consists of volunteers who
collaborate over the Net, each responsible for looking after a different
chunk of the system. These people have broken Linux down into a number
of packages, which are compressed files that can be downloaded to an
already functioning Debian Linux system, then opened up and unpacked
using a free installer application. Of course, as such, Debian has no
commercial arm---no distribution mechanism. You can download all Debian
packages over the Net, but most people will want to have them on a
CD-ROM. Several different companies have taken it upon themselves to
decoct all of the current Debian packages onto CD-ROMs and then sell
them. I buy mine from Linux Systems Labs. The cost for a three-disc set,
containing Debian in its entirety, is less than three dollars. But (and
this is an important distinction) not a single penny of that three
dollars is going to any of the coders who created Linux, nor to the
Debian packagers. It goes to Linux Systems Labs and it pays, not for the
software, or the packages, but for the cost of stamping out the CD-ROMs.

Every Linux distribution embodies some more or less clever hack for
circumventing the normal boot process and causing your computer, when it
is turned on, to organize itself, not as a PC running Windows, but as a
``host'' running Unix. This is slightly alarming the first time you see
it, but completely harmless. When a PC boots up, it goes through a
little self-test routine, taking an inventory of available disks and
memory, and then begins looking around for a disk to boot up from. In
any normal Windows computer that disk will be a hard drive. But if you
have your system configured right, it will look first for a floppy or
CD-ROM disk, and boot from that if one is available.

Linux exploits this chink in the defenses. Your computer notices a
bootable disk in the floppy or CD-ROM drive, loads in some object code
from that disk, and blindly begins to execute it. But this is not
Microsoft or Apple code, this is Linux code, and so at this point your
computer begins to behave very differently from what you are accustomed
to. Cryptic messages began to scroll up the screen. If you had booted a
commercial OS, you would, at this point, be seeing a ``Welcome to
MacOS'' cartoon, or a screen filled with clouds in a blue sky, and a
Windows logo. But under Linux you get a long telegram printed in stark
white letters on a black screen. There is no ``welcome!'' message. Most
of the telegram has the semi-inscrutable menace of graffiti tags.

\begin{lstlisting}
Dec 14 15:04:15 theRev syslogd 1.3-3#17: restart.
Dec 14 15:04:15 theRev kernel: klogd 1.3-3, log source = /proc/kmsg started.
Dec 14 15:04:15 theRev kernel: Loaded 3535 symbols from /System.map.
Dec 14 15:04:15 theRev kernel: Symbols match kernel version 2.0.30.
Dec 14 15:04:15 theRev kernel: No module symbols loaded.
Dec 14 15:04:15 theRev kernel: Intel MultiProcessor Specification v1.4 
Dec 14 15:04:15 theRev kernel: Virtual Wire compatibility mode.
Dec 14 15:04:15 theRev kernel: OEM ID: INTEL Product ID: 440FX APIC at: 0xFEE00000 
Dec 14 15:04:15 theRev kernel: Processor #0 Pentium(tm) Pro APIC version 17 
Dec 14 15:04:15 theRev kernel: Processor #1 Pentium(tm) Pro APIC version 17
Dec 14 15:04:15 theRev kernel: I/O APIC #2 Version 17 at 0xFEC00000.
Dec 14 15:04:15 theRev kernel: Processors: 2 Dec 14 15:04:15 theRev kernel: Console: 16 point font, 400 scans 
Dec 14 15:04:15 theRev kernel: Console: colour VGA+ 80x25, 1 virtual console (max 63) 
Dec 14 15:04:15 theRev kernel: pcibios_init : BIOS32 Service Directory structure at 0x000fdb70 
Dec 14 15:04:15 theRev kernel: pcibios_init : BIOS32 Service Directory entry at 0xfdb80 Dec 14 15:04:15 theRev kernel: pcibios_init : PCI BIOS revision 2.10 entry at 0xfdba1 
Dec 14 15:04:15 theRev kernel: Probing PCI hardware.
Dec 14 15:04:15 theRev kernel: Warning : Unknown PCI device (10b7:9001). Please read include/linux/pci.h 
\end{lstlisting}
The only parts of this that are readable, for normal people, are the
error messages and warnings. And yet it's noteworthy that Linux doesn't
stop, or crash, when it encounters an error; it spits out a pithy
complaint, gives up on whatever processes were damaged, and keeps on
rolling. This was decidedly not true of the early versions of Apple and
Microsoft OSes, for the simple reason that an OS that is not capable of
walking and chewing gum at the same time cannot possibly recover from
errors. Looking for, and dealing with, errors requires a separate
process running in parallel with the one that has erred. A kind of
superego, if you will, that keeps an eye on all of the others, and jumps
in when one goes astray. Now that MacOS and Windows can do more than one
thing at a time they are much better at dealing with errors than they
used to be, but they are not even close to Linux or other Unices in this
respect; and their greater complexity has made them vulnerable to new
types of errors.

\section{FALLIBILITY, ATONEMENT, REDEMPTION, TRUST, AND OTHER ARCANE
TECHNICAL CONCEPTS}

Linux is not capable of having any centrally organized policies
dictating how to write error messages and documentation, and so each
programmer writes his own. Usually they are in English even though tons
of Linux programmers are Europeans. Frequently they are funny. Always
they are honest. If something bad has happened because the software
simply isn't finished yet, or because the user screwed something up,
this will be stated forthrightly. The command line interface makes it
easy for programs to dribble out little comments, warnings, and messages
here and there. Even if the application is imploding like a damaged
submarine, it can still usually eke out a little S.O.S. message.
Sometimes when you finish working with a program and shut it down, you
find that it has left behind a series of mild warnings and low-grade
error messages in the command-line interface window from which you
launched it. As if the software were chatting to you about how it was
doing the whole time you were working with it.

Documentation, under Linux, comes in the form of man (short for manual)
pages. You can access these either through a GUI (xman) or from the
command line (man). Here is a sample from the man page for a program
called rsh:

``Stop signals stop the local rsh process only; this is arguably wrong,
but currently hard to fix for reasons too complicated to explain here.''

The man pages contain a lot of such material, which reads like the terse
mutterings of pilots wrestling with the controls of damaged airplanes.
The general feel is of a thousand monumental but obscure struggles seen
in the stop-action light of a strobe. Each programmer is dealing with
his own obstacles and bugs; he is too busy fixing them, and improving
the software, to explain things at great length or to maintain elaborate
pretensions.

In practice you hardly ever encounter a serious bug while running Linux.
When you do, it is almost always with commercial software (several
vendors sell software that runs under Linux). The operating system and
its fundamental utility programs are too important to contain serious
bugs. I have been running Linux every day since late 1995 and have seen
many application programs go down in flames, but I have never seen the
operating system crash. Never. Not once. There are quite a few Linux
systems that have been running continuously and working hard for months
or years without needing to be rebooted.

Commercial OSes have to adopt the same official stance towards errors as
Communist countries had towards poverty. For doctrinal reasons it was
not possible to admit that poverty was a serious problem in Communist
countries, because the whole point of Communism was to eradicate
poverty. Likewise, commercial OS companies like Apple and Microsoft
can't go around admitting that their software has bugs and that it
crashes all the time, any more than Disney can issue press releases
stating that Mickey Mouse is an actor in a suit.

This is a problem, because errors do exist and bugs do happen. Every few
months Bill Gates tries to demo a new Microsoft product in front of a
large audience only to have it blow up in his face. Commercial OS
vendors, as a direct consequence of being commercial, are forced to
adopt the grossly disingenuous position that bugs are rare aberrations,
usually someone else's fault, and therefore not really worth talking
about in any detail. This posture, which everyone knows to be absurd, is
not limited to press releases and ad campaigns. It informs the whole way
these companies do business and relate to their customers. If the
documentation were properly written, it would mention bugs, errors, and
crashes on every single page. If the on-line help systems that come with
these OSes reflected the experiences and concerns of their users, they
would largely be devoted to instructions on how to cope with crashes and
errors.

But this does not happen. Joint stock corporations are wonderful
inventions that have given us many excellent goods and services. They
are good at many things. Admitting failure is not one of them. Hell,
they can't even admit minor shortcomings.

Of course, this behavior is not as pathological in a corporation as it
would be in a human being. Most people, nowadays, understand that
corporate press releases are issued for the benefit of the corporation's
shareholders and not for the enlightenment of the public. Sometimes the
results of this institutional dishonesty can be dreadful, as with
tobacco and asbestos. In the case of commercial OS vendors it is nothing
of the kind, of course; it is merely annoying.

Some might argue that consumer annoyance, over time, builds up into a
kind of hardened plaque that can conceal serious decay, and that honesty
might therefore be the best policy in the long run; the jury is still
out on this in the operating system market. The business is expanding
fast enough that it's still much better to have billions of chronically
annoyed customers than millions of happy ones.

Most system administrators I know who work with Windows NT all the time
agree that when it hits a snag, it has to be re-booted, and when it gets
seriously messed up, the only way to fix it is to re-install the
operating system from scratch. Or at least this is the only way that
they know of to fix it, which amounts to the same thing. It is quite
possible that the engineers at Microsoft have all sorts of insider
knowledge on how to fix the system when it goes awry, but if they do,
they do not seem to be getting the message out to any of the actual
system administrators I know.

Because Linux is not commercial---because it is, in fact, free, as well
as rather difficult to obtain, install, and operate---it does not have
to maintain any pretensions as to its reliability. Consequently, it is
much more reliable. When something goes wrong with Linux, the error is
noticed and loudly discussed right away. Anyone with the requisite
technical knowledge can go straight to the source code and point out the
source of the error, which is then rapidly fixed by whichever hacker has
carved out responsibility for that particular program.

As far as I know, Debian is the only Linux distribution that has its own
constitution (\url{http://www.debian.org/devel/constitution}), but what
really sold me on it was its phenomenal bug database
(\url{http://www.debian.org/Bugs}), which is a sort of interactive
Doomsday Book of error, fallibility, and redemption. It is simplicity
itself. When had a problem with Debian in early January of 1997, I sent
in a message describing the problem to
\href{mailto:submit@bugs.debian.org}{\lstinline!submit@bugs.debian.org!}.
My problem was promptly assigned a bug report number (\#6518) and a
severity level (the available choices being critical, grave, important,
normal, fixed, and wishlist) and forwarded to mailing lists where Debian
people hang out. Within twenty-four hours I had received five e-mails
telling me how to fix the problem: two from North America, two from
Europe, and one from Australia. All of these e-mails gave me the same
suggestion, which worked, and made my problem go away. But at the same
time, a transcript of this exchange was posted to Debian's bug database,
so that if other users had the same problem later, they would be able to
search through and find the solution without having to enter a new,
redundant bug report.

Contrast this with the experience that I had when I tried to install
Windows NT 4.0 on the very same machine about ten months later, in late
1997. The installation program simply stopped in the middle with no
error messages. I went to the Microsoft Support website and tried to
perform a search for existing help documents that would address my
problem. The search engine was completely nonfunctional; it did nothing
at all. It did not even give me a message telling me that it was not
working.

Eventually I decided that my motherboard must be at fault; it was of a
slightly unusual make and model, and NT did not support as many
different motherboards as Linux. I am always looking for excuses, no
matter how feeble, to buy new hardware, so I bought a new motherboard
that was Windows NT logo-compatible, meaning that the Windows NT logo
was printed right on the box. I installed this into my computer and got
Linux running right away, then attempted to install Windows NT again.
Again, the installation died without any error message or explanation.
By this time a couple of weeks had gone by and I thought that perhaps
the search engine on the Microsoft Support website might be up and
running. I gave that a try but it still didn't work.

So I created a new Microsoft support account, then logged on to submit
the incident. I supplied my product ID number when asked, and then began
to follow the instructions on a series of help screens. In other words,
I was submitting a bug report just as with the Debian bug tracking
system. It's just that the interface was slicker---I was typing my
complaint into little text-editing boxes on Web forms, doing it all
through the GUI, whereas with Debian you send in an e-mail telegram. I
knew that when I was finished submitting the bug report, it would become
proprietary Microsoft information, and other users wouldn't be able to
see it. Many Linux users would refuse to participate in such a scheme on
ethical grounds, but I was willing to give it a shot as an experiment.
In the end, though I was never able to submit my bug report, because the
series of linked web pages that I was filling out eventually led me to a
completely blank page: a dead end.

So I went back and clicked on the buttons for ``phone support'' and
eventually was given a Microsoft telephone number. When I dialed this
number I got a series of piercing beeps and a recorded message from the
phone company saying ``We're sorry, your call cannot be completed as
dialed.''

I tried the search page again---it was still completely nonfunctional.
Then I tried PPI (Pay Per Incident) again. This led me through another
series of Web pages until I dead-ended at one reading: ``Notice-there is
no Web page matching your request.''

I tried it again, and eventually got to a Pay Per Incident screen
reading: ``OUT OF INCIDENTS. There are no unused incidents left in your
account. If you would like to purchase a support incident, click OK-you
will then be able to prepay for an incident\ldots{}.'' The cost per
incident was \$95.

The experiment was beginning to seem rather expensive, so I gave up on
the PPI approach and decided to have a go at the FAQs posted on
Microsoft's website. None of the available FAQs had anything to do with
my problem except for one entitled ``I am having some problems
installing NT'' which appeared to have been written by flacks, not
engineers.

So I gave up and still, to this day, have never gotten Windows NT
installed on that particular machine. For me, the path of least
resistance was simply to use Debian Linux.

In the world of open source software, bug reports are useful
information. Making them public is a service to other users, and
improves the OS. Making them public systematically is so important that
highly intelligent people voluntarily put time and money into running
bug databases. In the commercial OS world, however, reporting a bug is a
privilege that you have to pay lots of money for. But if you pay for it,
it follows that the bug report must be kept confidential---otherwise
anyone could get the benefit of your ninety-five bucks! And yet nothing
prevents NT users from setting up their own public bug database.

This is, in other words, another feature of the OS market that simply
makes no sense unless you view it in the context of culture. What
Microsoft is selling through Pay Per Incident isn't technical support so
much as the continued illusion that its customers are engaging in some
kind of rational business transaction. It is a sort of routine
maintenance fee for the upkeep of the fantasy. If people really wanted a
solid OS they would use Linux, and if they really wanted tech support
they would find a way to get it; Microsoft's customers want something
else.

As of this writing (Jan. 1999), something like 32,000 bugs have been
reported to the Debian Linux bug database. Almost all of them have been
fixed a long time ago. There are twelve ``critical'' bugs still
outstanding, of which the oldest was posted 79 days ago. There are 20
outstanding ``grave'' bugs of which the oldest is 1166 days old. There
are 48 ``important'' bugs and hundreds of ``normal'' and less important
ones.

Likewise, BeOS (which I'll get to in a minute) has its own bug database
(\url{http://www.be.com/developers/bugs/index.html}) with its own
classification system, including such categories as ``Not a Bug,''
``Acknowledged Feature,'' and ``Will Not Fix.'' Some of the ``bugs''
here are nothing more than Be hackers blowing off steam, and are
classified as ``Input Acknowledged.'' For example, I found one that was
posted on December 30th, 1998. It's in the middle of a long list of
bugs, wedged between one entitled ``Mouse working in very strange
fashion'' and another called ``Change of BView frame does not affect, if
BView not attached to a BWindow.''

This one is entitled

R4: BeOS missing megalomaniacal figurehead to harness and focus
developer rage

and it goes like this:

\begin{lstlisting}
----------------------------

Be Status: Input Acknowledged BeOS Version: R3.2 Component: unknown

Full Description:

The BeOS needs a megalomaniacal egomaniac sitting on its throne to give it a human character which everyone loves to hate. Without this, the BeOS will languish in the impersonifiable realm of OSs that people can never quite get a handle on. You can judge the success of an OS not by the quality of its features, but by how infamous and disliked the leaders behind them are.

I believe this is a side-effect of developer comraderie under miserable conditions. After all, misery loves company. I believe that making the BeOS less conceptually accessible and far less reliable will require developers to band together, thus developing the kind of community where strangers talk to one- another, kind of like at a grocery store before a huge snowstorm.

Following this same program, it will likely be necessary to move the BeOS headquarters to a far-less-comfortable climate. General environmental discomfort will breed this attitude within and there truly is no greater recipe for success. I would suggest Seattle, but I think it's already taken. You might try Washington, DC, but definitely not somewhere like San Diego or Tucson.

----------------------------
\end{lstlisting}
Unfortunately, the Be bug reporting system strips off the names of the
people who report the bugs (to protect them from retribution!?) and so I
don't know who wrote this.

So it would appear that I'm in the middle of crowing about the technical
and moral superiority of Debian Linux. But as almost always happens in
the OS world, it's more complicated than that. I have Windows NT running
on another machine, and the other day (Jan. 1999), when I had a problem
with it, I decided to have another go at Microsoft Support. This time
the search engine actually worked (though in order to reach it I had to
identify myself as ``advanced''). And instead of coughing up some
useless FAQ, it located about two hundred documents (I was using very
vague search criteria) that were obviously bug reports---though they
were called something else. Microsoft, in other words, has got a system
up and running that is functionally equivalent to Debian's bug database.
It looks and feels different, of course, but it contains technical
nitty-gritty and makes no bones about the existence of errors.

As I've explained, selling OSes for money is a basically untenable
position, and the only way Apple and Microsoft can get away with it is
by pursuing technological advancements as aggressively as they can, and
by getting people to believe in, and to pay for, a particular image: in
the case of Apple, that of the creative free thinker, and in the case of
Microsoft, that of the respectable techno-bourgeois. Just like Disney,
they're making money from selling an interface, a magic mirror. It has
to be polished and seamless or else the whole illusion is ruined and the
business plan vanishes like a mirage.

Accordingly, it was the case until recently that the people who wrote
manuals and created customer support websites for commercial OSes seemed
to have been barred, by their employers' legal or PR departments, from
admitting, even obliquely, that the software might contain bugs or that
the interface might be suffering from the blinking twelve problem. They
couldn't address users' actual difficulties. The manuals and websites
were therefore useless, and caused even technically self-assured users
to wonder whether they were going subtly insane.

When Apple engages in this sort of corporate behavior, one wants to
believe that they are really trying their best. We all want to give
Apple the benefit of the doubt, because mean old Bill Gates kicked the
crap out of them, and because they have good PR. But when Microsoft does
it, one almost cannot help becoming a paranoid conspiracist. Obviously
they are hiding something from us! And yet they are so powerful! They
are trying to drive us crazy!

This approach to dealing with one's customers was straight out of the
Central European totalitarianism of the mid-Twentieth Century. The
adjectives ``Kafkaesque'' and ``Orwellian'' come to mind. It couldn't
last, any more than the Berlin Wall could, and so now Microsoft has a
publicly available bug database. It's called something else, and it
takes a while to find it, but it's there.

They have, in other words, adapted to the two-tiered Eloi/Morlock
structure of technological society. If you're an Eloi you install
Windows, follow the instructions, hope for the best, and dumbly suffer
when it breaks. If you're a Morlock you go to the website, tell it that
you are ``advanced,'' find the bug database, and get the truth straight
from some anonymous Microsoft engineer.

But once Microsoft has taken this step, it raises the question, once
again, of whether there is any point to being in the OS business at all.
Customers might be willing to pay \$95 to report a problem to Microsoft
if, in return, they get some advice that no other user is getting. This
has the useful side effect of keeping the users alienated from one
another, which helps maintain the illusion that bugs are rare
aberrations. But once the results of those bug reports become openly
available on the Microsoft website, everything changes. No one is going
to cough up \$95 to report a problem when chances are good that some
other sucker will do it first, and that instructions on how to fix the
bug will then show up, for free, on a public website. And as the size of
the bug database grows, it eventually becomes an open admission, on
Microsoft's part, that their OSes have just as many bugs as their
competitors'. There is no shame in that; as I mentioned, Debian's bug
database has logged 32,000 reports so far. But it puts Microsoft on an
equal footing with the others and makes it a lot harder for their
customers---who want to believe---to believe.

\section{MEMENTO MORI}

Once the Linux machine has finished spitting out its jargonic opening
telegram, it prompts me to log in with a user name and a password. At
this point the machine is still running the command line interface, with
white letters on a black screen. There are no windows, menus, or
buttons. It does not respond to the mouse; it doesn't even know that the
mouse is there. It is still possible to run a lot of software at this
point. Emacs, for example, exists in both a CLI and a GUI version
(actually there are two GUI versions, reflecting some sort of doctrinal
schism between Richard Stallman and some hackers who got fed up with
him). The same is true of many other Unix programs. Many don't have a
GUI at all, and many that do are capable of running from the command
line.

Of course, since my computer only has one monitor screen, I can only see
one command line, and so you might think that I could only interact with
one program at a time. But if I hold down the Alt key and then hit the
F2 function button at the top of my keyboard, I am presented with a
fresh, blank, black screen with a login prompt at the top of it. I can
log in here and start some other program, then hit Alt-F1 and go back to
the first screen, which is still doing whatever it was when I left it.
Or I can do Alt-F3 and log in to a third screen, or a fourth, or a
fifth. On one of these screens I might be logged in as myself, on
another as root (the system administrator), on yet another I might be
logged on to some other computer over the Internet.

Each of these screens is called, in Unix-speak, a tty, which is an
abbreviation for teletype. So when I use my Linux system in this way I
am going right back to that small room at Ames High School where I first
wrote code twenty-five years ago, except that a tty is quieter and
faster than a teletype, and capable of running vastly superior software,
such as emacs or the GNU development tools.

It is easy (easy by Unix, not Apple/Microsoft standards) to configure a
Linux machine so that it will go directly into a GUI when you boot it
up. This way, you never see a tty screen at all. I still have mine boot
into the white-on-black teletype screen however, as a computational
memento mori. It used to be fashionable for a writer to keep a human
skull on his desk as a reminder that he was mortal, that all about him
was vanity. The tty screen reminds me that the same thing is true of
slick user interfaces.

The X Windows System, which is the GUI of Unix, has to be capable of
running on hundreds of different video cards with different chipsets,
amounts of onboard memory, and motherboard buses. Likewise, there are
hundreds of different types of monitors on the new and used market, each
with different specifications, and so there are probably upwards of a
million different possible combinations of card and monitor. The only
thing they all have in common is that they all work in VGA mode, which
is the old command-line screen that you see for a few seconds when you
launch Windows. So Linux always starts in VGA, with a teletype
interface, because at first it has no idea what sort of hardware is
attached to your computer. In order to get beyond the glass teletype and
into the GUI, you have to tell Linux exactly what kinds of hardware you
have. If you get it wrong, you'll get a blank screen at best, and at
worst you might actually destroy your monitor by feeding it signals it
can't handle.

When I started using Linux this had to be done by hand. I once spent the
better part of a month trying to get an oddball monitor to work for me,
and filled the better part of a composition book with increasingly
desperate scrawled notes. Nowadays, most Linux distributions ship with a
program that automatically scans the video card and self-configures the
system, so getting X Windows up and running is nearly as easy as
installing an Apple/Microsoft GUI. The crucial information goes into a
file (an ASCII text file, naturally) called XF86Config, which is worth
looking at even if your distribution creates it for you automatically.
For most people it looks like meaningless cryptic incantations, which is
the whole point of looking at it. An Apple/Microsoft system needs to
have the same information in order to launch its GUI, but it's apt to be
deeply hidden somewhere, and it's probably in a file that can't even be
opened and read by a text editor. All of the important files that make
Linux systems work are right out in the open. They are always ASCII text
files, so you don't need special tools to read them. You can look at
them any time you want, which is good, and you can mess them up and
render your system totally dysfunctional, which is not so good.

At any rate, assuming that my XF86Config file is just so, I enter the
command ``startx'' to launch the X Windows System. The screen blanks out
for a minute, the monitor makes strange twitching noises, then
reconstitutes itself as a blank gray desktop with a mouse cursor in the
middle. At the same time it is launching a window manager. X Windows is
pretty low-level software; it provides the infrastructure for a GUI, and
it's a heavy industrial infrastructure. But it doesn't do windows.
That's handled by another category of application that sits atop X
Windows, called a window manager. Several of these are available, all
free of course. The classic is twm (Tom's Window Manager) but there is a
smaller and supposedly more efficient variant of it called fvwm, which
is what I use. I have my eye on a completely different window manager
called Enlightenment, which may be the hippest single technology product
I have ever seen, in that (a) it is for Linux, (b) it is freeware, (c)
it is being developed by a very small number of obsessed hackers, and
(d) it looks amazingly cool; it is the sort of window manager that might
show up in the backdrop of an Aliens movie.

Anyway, the window manager acts as an intermediary between X Windows and
whatever software you want to use. It draws the window frames, menus,
and so on, while the applications themselves draw the actual content in
the windows. The applications might be of any sort: text editors, Web
browsers, graphics packages, or utility programs, such as a clock or
calculator. In other words, from this point on, you feel as if you have
been shunted into a parallel universe that is quite similar to the
familiar Apple or Microsoft one, but slightly and pervasively different.
The premier graphics program under Apple/Microsoft is Adobe Photoshop,
but under Linux it's something called The GIMP. Instead of the Microsoft
Office Suite, you can buy something called ApplixWare. Many commercial
software packages, such as Mathematica, Netscape Communicator, and Adobe
Acrobat, are available in Linux versions, and depending on how you set
up your window manager you can make them look and behave just as they
would under MacOS or Windows.

But there is one type of window you'll see on Linux GUI that is rare or
nonexistent under other OSes. These windows are called ``xterm'' and
contain nothing but lines of text---this time, black text on a white
background, though you can make them be different colors if you choose.
Each xterm window is a separate command line interface---a tty in a
window. So even when you are in full GUI mode, you can still talk to
your Linux machine through a command-line interface.

There are many good pieces of Unix software that do not have GUIs at
all. This might be because they were developed before X Windows was
available, or because the people who wrote them did not want to suffer
through all the hassle of creating a GUI, or because they simply do not
need one. In any event, those programs can be invoked by typing their
names into the command line of an xterm window. The whoami command,
mentioned earlier, is a good example. There is another called wc (``word
count'') which simply returns the number of lines, words, and characters
in a text file.

The ability to run these little utility programs on the command line is
a great virtue of Unix, and one that is unlikely to be duplicated by
pure GUI operating systems. The wc command, for example, is the sort of
thing that is easy to write with a command line interface. It probably
does not consist of more than a few lines of code, and a clever
programmer could probably write it in a single line. In compiled form it
takes up just a few bytes of disk space. But the code required to give
the same program a graphical user interface would probably run into
hundreds or even thousands of lines, depending on how fancy the
programmer wanted to make it. Compiled into a runnable piece of
software, it would have a large overhead of GUI code. It would be slow
to launch and it would use up a lot of memory. This would simply not be
worth the effort, and so ``wc'' would never be written as an independent
program at all. Instead users would have to wait for a word count
feature to appear in a commercial software package.

GUIs tend to impose a large overhead on every single piece of software,
even the smallest, and this overhead completely changes the programming
environment. Small utility programs are no longer worth writing. Their
functions, instead, tend to get swallowed up into omnibus software
packages. As GUIs get more complex, and impose more and more overhead,
this tendency becomes more pervasive, and the software packages grow
ever more colossal; after a point they begin to merge with each other,
as Microsoft Word and Excel and PowerPoint have merged into Microsoft
Office: a stupendous software Wal-Mart sitting on the edge of a town
filled with tiny shops that are all boarded up.

It is an unfair analogy, because when a tiny shop gets boarded up it
means that some small shopkeeper has lost his business. Of course
nothing of the kind happens when ``wc'' becomes subsumed into one of
Microsoft Word's countless menu items. The only real drawback is a loss
of flexibility for the user, but it is a loss that most customers
obviously do not notice or care about. The most serious drawback to the
Wal-Mart approach is that most users only want or need a tiny fraction
of what is contained in these giant software packages. The remainder is
clutter, dead weight. And yet the user in the next cubicle over will
have completely different opinions as to what is useful and what isn't.

The other important thing to mention, here, is that Microsoft has
included a genuinely cool feature in the Office package: a Basic
programming package. Basic is the first computer language that I
learned, back when I was using the paper tape and the teletype. By using
the version of Basic that comes with Office you can write your own
little utility programs that know how to interact with all of the little
doohickeys, gewgaws, bells, and whistles in Office. Basic is easier to
use than the languages typically employed in Unix command-line
programming, and Office has reached many, many more people than the GNU
tools. And so it is quite possible that this feature of Office will, in
the end, spawn more hacking than GNU.

But now I'm talking about application software, not operating systems.
And as I've said, Microsoft's application software tends to be very good
stuff. I don't use it very much, because I am nowhere near their target
market. If Microsoft ever makes a software package that I use and like,
then it really will be time to dump their stock, because I am a market
segment of one.

\section{GEEK FATIGUE}

Over the years that I've been working with Linux I have filled three and
a half notebooks logging my experiences. I only begin writing things
down when I'm doing something complicated, like setting up X Windows or
fooling around with my Internet connection, and so these notebooks
contain only the record of my struggles and frustrations. When things
are going well for me, I'll work along happily for many months without
jotting down a single note. So these notebooks make for pretty bleak
reading. Changing anything under Linux is a matter of opening up various
of those little ASCII text files and changing a word here and a
character there, in ways that are extremely significant to how the
system operates.

Many of the files that control how Linux operates are nothing more than
command lines that became so long and complicated that not even Linux
hackers could type them correctly. When working with something as
powerful as Linux, you can easily devote a full half-hour to engineering
a single command line. For example, the ``find'' command, which searches
your file system for files that match certain criteria, is fantastically
powerful and general. Its ``man'' is eleven pages long, and these are
pithy pages; you could easily expand them into a whole book. And if that
is not complicated enough in and of itself, you can always pipe the
output of one Unix command to the input of another, equally complicated
one. The ``pon'' command, which is used to fire up a PPP connection to
the Internet, requires so much detailed information that it is basically
impossible to launch it entirely from the command line. Instead you
abstract big chunks of its input into three or four different files. You
need a dialing script, which is effectively a little program telling it
how to dial the phone and respond to various events; an options file,
which lists up to about sixty different options on how the PPP
connection is to be set up; and a secrets file, giving information about
your password.

Presumably there are godlike Unix hackers somewhere in the world who
don't need to use these little scripts and options files as crutches,
and who can simply pound out fantastically complex command lines without
making typographical errors and without having to spend hours flipping
through documentation. But I'm not one of them. Like almost all Linux
users, I depend on having all of those details hidden away in thousands
of little ASCII text files, which are in turn wedged into the recesses
of the Unix filesystem. When I want to change something about the way my
system works, I edit those files. I know that if I don't keep track of
every little change I've made, I won't be able to get your system back
in working order after I've gotten it all messed up. Keeping
hand-written logs is tedious, not to mention kind of anachronistic. But
it's necessary.

I probably could have saved myself a lot of headaches by doing business
with a company called Cygnus Support, which exists to provide assistance
to users of free software. But I didn't, because I wanted to see if I
could do it myself. The answer turned out to be yes, but just barely.
And there are many tweaks and optimizations that I could probably make
in my system that I have never gotten around to attempting, partly
because I get tired of being a Morlock some days, and partly because I
am afraid of fouling up a system that generally works well.

Though Linux works for me and many other users, its sheer power and
generality is its Achilles' heel. If you know what you are doing, you
can buy a cheap PC from any computer store, throw away the Windows discs
that come with it, turn it into a Linux system of mind-boggling
complexity and power. You can hook it up to twelve other Linux boxes and
make it into part of a parallel computer. You can configure it so that a
hundred different people can be logged onto it at once over the
Internet, via as many modem lines, Ethernet cards, TCP/IP sockets, and
packet radio links. You can hang half a dozen different monitors off of
it and play DOOM with someone in Australia while tracking communications
satellites in orbit and controlling your house's lights and thermostats
and streaming live video from your web-cam and surfing the Net and
designing circuit boards on the other screens. But the sheer power and
complexity of the system---the qualities that make it so vastly
technically superior to other OSes---sometimes make it seem too
formidable for routine day-to-day use.

Sometimes, in other words, I just want to go to Disneyland.

The ideal OS for me would be one that had a well-designed GUI that was
easy to set up and use, but that included terminal windows where I could
revert to the command line interface, and run GNU software, when it made
sense. A few years ago, Be Inc. invented exactly that OS. It is called
the BeOS.

\section{ETRE}

Many people in the computer business have had a difficult time grappling
with Be, Incorporated, for the simple reason that nothing about it seems
to make any sense whatsoever. It was launched in late 1990, which makes
it roughly contemporary with Linux. From the beginning it has been
devoted to creating a new operating system that is, by design,
incompatible with all the others (though, as we shall see, it is
compatible with Unix in some very important ways). If a definition of
``celebrity'' is someone who is famous for being famous, then Be is an
anti-celebrity. It is famous for not being famous; it is famous for
being doomed. But it has been doomed for an awfully long time.

Be's mission might make more sense to hackers than to other people. In
order to explain why I need to explain the concept of cruft, which, to
people who write code, is nearly as abhorrent as unnecessary repetition.

If you've been to San Francisco you may have seen older buildings that
have undergone ``seismic upgrades,'' which frequently means that
grotesque superstructures of modern steelwork are erected around
buildings made in, say, a Classical style. When new threats arrive---if
we have an Ice Age, for example---additional layers of even more
high-tech stuff may be constructed, in turn, around these, until the
original building is like a holy relic in a cathedral---a shard of
yellowed bone enshrined in half a ton of fancy protective junk.

Analogous measures can be taken to keep creaky old operating systems
working. It happens all the time. Ditching an worn-out old OS ought to
be simplified by the fact that, unlike old buildings, OSes have no
aesthetic or cultural merit that makes them intrinsically worth saving.
But it doesn't work that way in practice. If you work with a computer,
you have probably customized your ``desktop,'' the environment in which
you sit down to work every day, and spent a lot of money on software
that works in that environment, and devoted much time to familiarizing
yourself with how it all works. This takes a lot of time, and time is
money. As already mentioned, the desire to have one's interactions with
complex technologies simplified through the interface, and to surround
yourself with virtual tchotchkes and lawn ornaments, is natural and
pervasive---presumably a reaction against the complexity and formidable
abstraction of the computer world. Computers give us more choices than
we really want. We prefer to make those choices once, or accept the
defaults handed to us by software companies, and let sleeping dogs lie.
But when an OS gets changed, all the dogs jump up and start barking.

The average computer user is a technological antiquarian who doesn't
really like things to change. He or she is like an urban professional
who has just bought a charming fixer-upper and is now moving the
furniture and knicknacks around, and reorganizing the kitchen cupboards,
so that everything's just right. If it is necessary for a bunch of
engineers to scurry around in the basement shoring up the foundation so
that it can support the new cast-iron claw-foot bathtub, and snaking new
wires and pipes through the walls to supply modern appliances, why, so
be it---engineers are cheap, at least when millions of OS users split
the cost of their services.

Likewise, computer users want to have the latest Pentium in their
machines, and to be able to surf the web, without messing up all the
stuff that makes them feel as if they know what the hell is going on.
Sometimes this is actually possible. Adding more RAM to your system is a
good example of an upgrade that is not likely to screw anything up.

Alas, very few upgrades are this clean and simple. Lawrence Lessig, the
whilom Special Master in the Justice Department's antitrust suit against
Microsoft, complained that he had installed Internet Explorer on his
computer, and in so doing, lost all of his bookmarks---his personal list
of signposts that he used to navigate through the maze of the Internet.
It was as if he'd bought a new set of tires for his car, and then, when
pulling away from the garage, discovered that, owing to some inscrutable
side-effect, every signpost and road map in the world had been
destroyed. If he's like most of us, he had put a lot of work into
compiling that list of bookmarks. This is only a small taste of the sort
of trouble that upgrades can cause. Crappy old OSes have value in the
basically negative sense that changing to new ones makes us wish we'd
never been born.

All of the fixing and patching that engineers must do in order to give
us the benefits of new technology without forcing us to think about it,
or to change our ways, produces a lot of code that, over time, turns
into a giant clot of bubble gum, spackle, baling wire and duct tape
surrounding every operating system. In the jargon of hackers, it is
called ``cruft.'' An operating system that has many, many layers of it
is described as ``crufty.'' Hackers hate to do things twice, but when
they see something crufty, their first impulse is to rip it out, throw
it away, and start anew.

If Mark Twain were brought back to San Francisco today and dropped into
one of these old seismically upgraded buildings, it would look just the
same to him, with all the doors and windows in the same places---but if
he stepped outside, he wouldn't recognize it. And---if he'd been brought
back with his wits intact---he might question whether the building had
been worth going to so much trouble to save. At some point, one must ask
the question: is this really worth it, or should we maybe just tear it
down and put up a good one? Should we throw another human wave of
structural engineers at stabilizing the Leaning Tower of Pisa, or should
we just let the damn thing fall over and build a tower that doesn't
suck?

Like an upgrade to an old building, cruft always seems like a good idea
when the first layers of it go on---just routine maintenance, sound
prudent management. This is especially true if (as it were) you never
look into the cellar, or behind the drywall. But if you are a hacker who
spends all his time looking at it from that point of view, cruft is
fundamentally disgusting, and you can't avoid wanting to go after it
with a crowbar. Or, better yet, simply walk out of the building---let
the Leaning Tower of Pisa fall over---and go make a new one THAT DOESN'T
LEAN.

For a long time it was obvious to Apple, Microsoft, and their customers
that the first generation of GUI operating systems was doomed, and that
they would eventually need to be ditched and replaced with completely
fresh ones. During the late Eighties and early Nineties, Apple launched
a few abortive efforts to make fundamentally new post-Mac OSes such as
Pink and Taligent. When those efforts failed they launched a new project
called Copland which also failed. In 1997 they flirted with the idea of
acquiring Be, but instead they acquired Next, which has an OS called
NextStep that is, in effect, a variant of Unix. As these efforts went
on, and on, and on, and failed and failed and failed, Apple's engineers,
who were among the best in the business, kept layering on the cruft.
They were gamely trying to turn the little toaster into a multi-tasking,
Internet-savvy machine, and did an amazingly good job of it for a
while---sort of like a movie hero running across a jungle river by
hopping across crocodiles' backs. But in the real world you eventually
run out of crocodiles, or step on a really smart one.

Speaking of which, Microsoft tackled the same problem in a considerably
more orderly way by creating a new OS called Windows NT, which is
explicitly intended to be a direct competitor of Unix. NT stands for
``New Technology'' which might be read as an explicit rejection of
cruft. And indeed, NT is reputed to be a lot less crufty than what MacOS
eventually turned into; at one point the documentation needed to write
code on the Mac filled something like 24 binders. Windows 95 was, and
Windows 98 is, crufty because they have to be backward-compatible with
older Microsoft OSes. Linux deals with the cruft problem in the same way
that Eskimos supposedly dealt with senior citizens: if you insist on
using old versions of Linux software, you will sooner or later find
yourself drifting through the Bering Straits on a dwindling ice floe.
They can get away with this because most of the software is free, so it
costs nothing to download up-to-date versions, and because most Linux
users are Morlocks.

The great idea behind BeOS was to start from a clean sheet of paper and
design an OS the right way. And that is exactly what they did. This was
obviously a good idea from an aesthetic standpoint, but does not a sound
business plan make. Some people I know in the GNU/Linux world are
annoyed with Be for going off on this quixotic adventure when their
formidable skills could have been put to work helping to promulgate
Linux.

Indeed, none of it makes sense until you remember that the founder of
the company, Jean-Louis Gassee, is from France---a country that for many
years maintained its own separate and independent version of the English
monarchy at a court in St.~Germaines, complete with courtiers,
coronation ceremonies, a state religion and a foreign policy. Now, the
same annoying yet admirable stiff-neckedness that gave us the Jacobites,
the force de frappe, Airbus, and ARRET signs in Quebec, has brought us a
really cool operating system. I fart in your general direction,
Anglo-Saxon pig-dogs!

To create an entirely new OS from scratch, just because none of the
existing ones was exactly right, struck me as an act of such colossal
nerve that I felt compelled to support it. I bought a BeBox as soon as I
could. The BeBox was a dual-processor machine, powered by Motorola
chips, made specifically to run the BeOS; it could not run any other
operating system. That's why I bought it. I felt it was a way to burn my
bridges. Its most distinctive feature is two columns of LEDs on the
front panel that zip up and down like tachometers to convey a sense of
how hard each processor is working. I thought it looked cool, and
besides, I reckoned that when the company went out of business in a few
months, my BeBox would be a valuable collector's item.

Now it is about two years later and I am typing this on my BeBox. The
LEDs (Das Blinkenlights, as they are called in the Be community) flash
merrily next to my right elbow as I hit the keys. Be, Inc. is still in
business, though they stopped making BeBoxes almost immediately after I
bought mine. They made the sad, but probably quite wise decision that
hardware was a sucker's game, and ported the BeOS to Macintoshes and Mac
clones. Since these used the same sort of Motorola chips that powered
the BeBox, this wasn't especially hard.

Very soon afterwards, Apple strangled the Mac-clone makers and restored
its hardware monopoly. So, for a while, the only new machines that could
run BeOS were made by Apple.

By this point Be, like Spiderman with his Spider-sense, had developed a
keen sense of when they were about to get crushed like a bug. Even if
they hadn't, the notion of being dependent on Apple---so frail and yet
so vicious---for their continued existence should have put a fright into
anyone. Now engaged in their own crocodile-hopping adventure, they
ported the BeOS to Intel chips---the same chips used in Windows
machines. And not a moment too soon, for when Apple came out with its
new top-of-the-line hardware, based on the Motorola G3 chip, they
withheld the technical data that Be's engineers would need to make the
BeOS run on those machines. This would have killed Be, just like a slug
between the eyes, if they hadn't made the jump to Intel.

So now BeOS runs on an assortment of hardware that is almost incredibly
motley: BeBoxes, aging Macs and Mac orphan-clones, and Intel machines
that are intended to be used for Windows. Of course the latter type are
ubiquitous and shockingly cheap nowadays, so it would appear that Be's
hardware troubles are finally over. Some German hackers have even come
up with a Das Blinkenlights replacement: it's a circuit board kit that
you can plug into PC-compatible machines running BeOS. It gives you the
zooming LED tachometers that were such a popular feature of the BeBox.

My BeBox is already showing its age, as all computers do after a couple
of years, and sooner or later I'll probably have to replace it with an
Intel machine. Even after that, though, I will still be able to use it.
Because, inevitably, someone has now ported Linux to the BeBox.

At any rate, BeOS has an extremely well-thought-out GUI built on a
technological framework that is solid. It is based from the ground up on
modern object-oriented software principles. BeOS software consists of
quasi-independent software entities called objects, which communicate by
sending messages to each other. The OS itself is made up of such
objects, and serves as a kind of post office or Internet that routes
messages to and fro, from object to object. The OS is multi-threaded,
which means that like all other modern OSes it can walk and chew gum at
the same time; but it gives programmers a lot of power over spawning and
terminating threads, or independent sub-processes. It is also a
multi-processing OS, which means that it is inherently good at running
on computers that have more than one CPU (Linux and Windows NT can also
do this proficiently).

For this user, a big selling point of BeOS is the built-in Terminal
application, which enables you to open up windows that are equivalent to
the xterm windows in Linux. In other words, the command line interface
is available if you want it. And because BeOS hews to a certain standard
called POSIX, it is capable of running most of the GNU software. That is
to say that the vast array of command-line software developed by the GNU
crowd will work in BeOS terminal windows without complaint. This
includes the GNU development tools-the compiler and linker. And it
includes all of the handy little utility programs. I'm writing this
using a modern sort of user-friendly text editor called Pe, written by a
Dutchman named Maarten Hekkelman, but when I want to find out how long
it is, I jump to a terminal window and run ``wc.''

As is suggested by the sample bug report I quoted earlier, people who
work for Be, and developers who write code for BeOS, seem to be enjoying
themselves more than their counterparts in other OSes. They also seem to
be a more diverse lot in general. A couple of years ago I went to an
auditorium at a local university to see some representatives of Be put
on a dog-and-pony show. I went because I assumed that the place would be
empty and echoing, and I felt that they deserved an audience of at least
one. In fact, I ended up standing in an aisle, for hundreds of students
had packed the place. It was like a rock concert. One of the two Be
engineers on the stage was a black man, which unfortunately is a very
odd thing in the high-tech world. The other made a ringing denunciation
of cruft, and extolled BeOS for its cruft-free qualities, and actually
came out and said that in ten or fifteen years, when BeOS had become all
crufty like MacOS and Windows 95, it would be time to simply throw it
away and create a new OS from scratch. I doubt that this is an official
Be, Inc. policy, but it sure made a big impression on everyone in the
room! During the late Eighties, the MacOS was, for a time, the OS of
cool people-artists and creative-minded hackers-and BeOS seems to have
the potential to attract the same crowd now. Be mailing lists are
crowded with hackers with names like Vladimir and Olaf and Pierre,
sending flames to each other in fractured techno-English.

The only real question about BeOS is whether or not it is doomed.

Of late, Be has responded to the tiresome accusation that they are
doomed with the assertion that BeOS is ``a media operating system'' made
for media content creators, and hence is not really in competition with
Windows at all. This is a little bit disingenuous. To go back to the car
dealership analogy, it is like the Batmobile dealer claiming that he is
not really in competition with the others because his car can go three
times as fast as theirs and is also capable of flying.

Be has an office in Paris, and, as mentioned, the conversation on Be
mailing lists has a strongly European flavor. At the same time they have
made strenuous efforts to find a niche in Japan, and Hitachi has
recently begun bundling BeOS with their PCs. So if I had to make wild
guess I'd say that they are playing Go while Microsoft is playing chess.
They are staying clear, for now, of Microsoft's overwhelmingly strong
position in North America. They are trying to get themselves established
around the edges of the board, as it were, in Europe and Japan, where
people may be more open to alternative OSes, or at least more hostile to
Microsoft, than they are in the United States.

What holds Be back in this country is that the smart people are afraid
to look like suckers. You run the risk of looking naive when you say
``I've tried the BeOS and here's what I think of it.'' It seems much
more sophisticated to say ``Be's chances of carving out a new niche in
the highly competitive OS market are close to nil.''

It is, in techno-speak, a problem of mindshare. And in the OS business,
mindshare is more than just a PR issue; it has direct effects on the
technology itself. All of the peripheral gizmos that can be hung off of
a personal computer---the printers, scanners, PalmPilot interfaces, and
Lego Mindstorms---require pieces of software called drivers. Likewise,
video cards and (to a lesser extent) monitors need drivers. Even the
different types of motherboards on the market relate to the OS in
different ways, and separate code is required for each one. All of this
hardware-specific code must not only written but also tested, debugged,
upgraded, maintained, and supported. Because the hardware market has
become so vast and complicated, what really determines an OS's fate is
not how good the OS is technically, or how much it costs, but rather the
availability of hardware-specific code. Linux hackers have to write that
code themselves, and they have done an amazingly good job of keeping up
to speed. Be, Inc. has to write all their own drivers, though as BeOS
has begun gathering momentum, third-party developers have begun to
contribute drivers, which are available on Be's web site.

But Microsoft owns the high ground at the moment, because it doesn't
have to write its own drivers. Any hardware maker bringing a new video
card or peripheral device to market today knows that it will be
unsalable unless it comes with the hardware-specific code that will make
it work under Windows, and so each hardware maker has accepted the
burden of creating and maintaining its own library of drivers.

\section{MINDSHARE}

The U.S. Government's assertion that Microsoft has a monopoly in the OS
market might be the most patently absurd claim ever advanced by the
legal mind. Linux, a technically superior operating system, is being
given away for free, and BeOS is available at a nominal price. This is
simply a fact, which has to be accepted whether or not you like
Microsoft.

Microsoft is really big and rich, and if some of the government's
witnesses are to be believed, they are not nice guys. But the accusation
of a monopoly simply does not make any sense.

What is really going on is that Microsoft has seized, for the time
being, a certain type of high ground: they dominate in the competition
for mindshare, and so any hardware or software maker who wants to be
taken seriously feels compelled to make a product that is compatible
with their operating systems. Since Windows-compatible drivers get
written by the hardware makers, Microsoft doesn't have to write them; in
effect, the hardware makers are adding new components to Windows, making
it a more capable OS, without charging Microsoft for the service. It is
a very good position to be in. The only way to fight such an opponent is
to have an army of highly competetent coders who write equivalent
drivers for free, which Linux does.

But possession of this psychological high ground is different from a
monopoly in any normal sense of that word, because here the dominance
has nothing to do with technical performance or price. The old
robber-baron monopolies were monopolies because they physically
controlled means of production and/or distribution. But in the software
business, the means of production is hackers typing code, and the means
of distribution is the Internet, and no one is claiming that Microsoft
controls those.

Here, instead, the dominance is inside the minds of people who buy
software. Microsoft has power because people believe it does. This power
is very real. It makes lots of money. Judging from recent legal
proceedings in both Washingtons, it would appear that this power and
this money have inspired some very peculiar executives to come out and
work for Microsoft, and that Bill Gates should have administered saliva
tests to some of them before issuing them Microsoft ID cards.

But this is not the sort of power that fits any normal definition of the
word ``monopoly,'' and it's not amenable to a legal fix. The courts may
order Microsoft to do things differently. They might even split the
company up. But they can't really do anything about a mindshare
monopoly, short of taking every man, woman, and child in the developed
world and subjecting them to a lengthy brainwashing procedure.

Mindshare dominance is, in other words, a really odd sort of beast,
something that the framers of our antitrust laws couldn't possibly have
imagined. It looks like one of these modern, wacky chaos-theory
phenomena, a complexity thing, in which a whole lot of independent but
connected entities (the world's computer users), making decisions on
their own, according to a few simple rules of thumb, generate a large
phenomenon (total domination of the market by one company) that cannot
be made sense of through any kind of rational analysis. Such phenomena
are fraught with concealed tipping-points and all a-tangle with bizarre
feedback loops, and cannot be understood; people who try, end up (a)
going crazy, (b) giving up, (c) forming crackpot theories, or (d)
becoming high-paid chaos theory consultants.

Now, there might be one or two people at Microsoft who are dense enough
to believe that mindshare dominance is some kind of stable and enduring
position. Maybe that even accounts for some of the weirdos they've hired
in the pure-business end of the operation, the zealots who keep getting
hauled into court by enraged judges. But most of them must have the wit
to understand that phenomena like these are maddeningly unstable, and
that there's no telling what weird, seemingly inconsequential event
might cause the system to shift into a radically different
configuration.

To put it another way, Microsoft can be confident that Thomas Penfield
Jackson will not hand down an order that the brains of everyone in the
developed world are to be summarily re-programmed. But there's no way to
predict when people will decide, en masse, to re-program their own
brains. This might explain some of Microsoft's behavior, such as their
policy of keeping eerily large reserves of cash sitting around, and the
extreme anxiety that they display whenever something like Java comes
along.

I have never seen the inside of the building at Microsoft where the top
executives hang out, but I have this fantasy that in the hallways, at
regular intervals, big red alarm boxes are bolted to the wall. Each
contains a large red button protected by a windowpane. A metal hammer
dangles on a chain next to it. Above is a big sign reading: IN THE EVENT
OF A CRASH IN MARKET SHARE, BREAK GLASS.

What happens when someone shatters the glass and hits the button, I
don't know, but it sure would be interesting to find out. One imagines
banks collapsing all over the world as Microsoft withdraws its cash
reserves, and shrink-wrapped pallet-loads of hundred-dollar bills
dropping from the skies. No doubt, Microsoft has a plan. But what I
would really like to know is whether, at some level, their programmers
might heave a big sigh of relief if the burden of writing the One
Universal Interface to Everything were suddenly lifted from their
shoulders.

\section{THE RIGHT PINKY OF GOD}

In his book The Life of the Cosmos, which everyone should read, Lee
Smolin gives the best description I've ever read of how our universe
emerged from an uncannily precise balancing of different fundamental
constants. The mass of the proton, the strength of gravity, the range of
the weak nuclear force, and a few dozen other fundamental constants
completely determine what sort of universe will emerge from a Big Bang.
If these values had been even slightly different, the universe would
have been a vast ocean of tepid gas or a hot knot of plasma or some
other basically uninteresting thing---a dud, in other words. The only
way to get a universe that's not a dud---that has stars, heavy elements,
planets, and life---is to get the basic numbers just right. If there
were some machine, somewhere, that could spit out universes with
randomly chosen values for their fundamental constants, then for every
universe like ours it would produce 10\^{}229 duds.

Though I haven't sat down and run the numbers on it, to me this seems
comparable to the probability of making a Unix computer do something
useful by logging into a tty and typing in command lines when you have
forgotten all of the little options and keywords. Every time your right
pinky slams that ENTER key, you are making another try. In some cases
the operating system does nothing. In other cases it wipes out all of
your files. In most cases it just gives you an error message. In other
words, you get many duds. But sometimes, if you have it all just right,
the computer grinds away for a while and then produces something like
emacs. It actually generates complexity, which is Smolin's criterion for
interestingness.

Not only that, but it's beginning to look as if, once you get below a
certain size---way below the level of quarks, down into the realm of
string theory---the universe can't be described very well by physics as
it has been practiced since the days of Newton. If you look at a small
enough scale, you see processes that look almost computational in
nature.

I think that the message is very clear here: somewhere outside of and
beyond our universe is an operating system, coded up over incalculable
spans of time by some kind of hacker-demiurge. The cosmic operating
system uses a command-line interface. It runs on something like a
teletype, with lots of noise and heat; punched-out bits flutter down
into its hopper like drifting stars. The demiurge sits at his teletype,
pounding out one command line after another, specifying the values of
fundamental constants of physics:

\begin{lstlisting}
universe -G 6.672e-11 -e 1.602e-19 -h 6.626e-34 -protonmass 1.673e-27....
\end{lstlisting}
and when he's finished typing out the command line, his right pinky
hesitates above the ENTER key for an aeon or two, wondering what's going
to happen; then down it comes---and the WHACK you hear is another Big
Bang.

Now THAT is a cool operating system, and if such a thing were actually
made available on the Internet (for free, of course) every hacker in the
world would download it right away and then stay up all night long
messing with it, spitting out universes right and left. Most of them
would be pretty dull universes but some of them would be simply amazing.
Because what those hackers would be aiming for would be much more
ambitious than a universe that had a few stars and galaxies in it. Any
run-of-the-mill hacker would be able to do that. No, the way to gain a
towering reputation on the Internet would be to get so good at tweaking
your command line that your universes would spontaneously develop life.
And once the way to do that became common knowledge, those hackers would
move on, trying to make their universes develop the right kind of life,
trying to find the one change in the Nth decimal place of some physical
constant that would give us an Earth in which, say, Hitler had been
accepted into art school after all, and had ended up his days as a
street artist with cranky political opinions.

Even if that fantasy came true, though, most users (including myself, on
certain days) wouldn't want to bother learning to use all of those
arcane commands, and struggling with all of the failures; a few dud
universes can really clutter up your basement. After we'd spent a while
pounding out command lines and hitting that ENTER key and spawning dull,
failed universes, we would start to long for an OS that would go all the
way to the opposite extreme: an OS that had the power to do
everything---to live our life for us. In this OS, all of the possible
decisions we could ever want to make would have been anticipated by
clever programmers, and condensed into a series of dialog boxes. By
clicking on radio buttons we could choose from among mutually exclusive
choices (HETEROSEXUAL/HOMOSEXUAL). Columns of check boxes would enable
us to select the things that we wanted in our life (GET MARRIED/WRITE
GREAT AMERICAN NOVEL) and for more complicated options we could fill in
little text boxes (NUMBER OF DAUGHTERS: NUMBER OF SONS:).

Even this user interface would begin to look awfully complicated after a
while, with so many choices, and so many hidden interactions between
choices. It could become damn near unmanageable---the blinking twelve
problem all over again. The people who brought us this operating system
would have to provide templates and wizards, giving us a few default
lives that we could use as starting places for designing our own.
Chances are that these default lives would actually look pretty damn
good to most people, good enough, anyway, that they'd be reluctant to
tear them open and mess around with them for fear of making them worse.
So after a few releases the software would begin to look even simpler:
you would boot it up and it would present you with a dialog box with a
single large button in the middle labeled: LIVE. Once you had clicked
that button, your life would begin. If anything got out of whack, or
failed to meet your expectations, you could complain about it to
Microsoft's Customer Support Department. If you got a flack on the line,
he or she would tell you that your life was actually fine, that there
was not a thing wrong with it, and in any event it would be a lot better
after the next upgrade was rolled out. But if you persisted, and
identified yourself as Advanced, you might get through to an actual
engineer.

What would the engineer say, after you had explained your problem, and
enumerated all of the dissatisfactions in your life? He would probably
tell you that life is a very hard and complicated thing; that no
interface can change that; that anyone who believes otherwise is a
sucker; and that if you don't like having choices made for you, you
should start making your own.

\end{document}
